\chapter{Preliminaries}
\label{cha:1}
A chapter is a logical unit. It normally starts with an introduction, which
you are reading now. The last topic of the chapter holds the conclusion.

\section{Notation}
First comes the introduction to this topic.

\lipsum[55]

\section{Group Theory}
\lipsum[33]

\section{Bilinear Pairings}
\lipsum[64]

\subsection{Definition}
\lipsum[56-57]

\subsection{Bilinear Diffie-Hellman Assumption}
\lipsum[58]

\subsection{Variants of the Bilinear Diffie-Hellman Assumption}
\lipsum[59]

\section{Commitment Schemes}
\lipsum[60]

\subsection{Definition}
\lipsum[61]

\subsection{Pedersen Commitment Scheme}
\lipsum[62]

\section{Proofs of Knowledge}
\lipsum[63]

\subsection{Definition}
\lipsum[64]

\subsection{Zero-knowledge Proof}
\lipsum[65]

\subsection{Security Model}
\lipsum[66]

\subsubsection{Standard Model}
\lipsum[67]

\subsubsection{Random Oracle Assumption}
\lipsum[68]

\section{Conclusion}
The final section of the chapter gives an overview of the important results
of this chapter. This implies that the introductory chapter and the
concluding chapter don't need a conclusion.

\lipsum[66]

%%% Local Variables: 
%%% mode: latex
%%% TeX-master: "thesis"
%%% End: 
