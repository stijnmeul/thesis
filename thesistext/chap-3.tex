\chapter{Design of a Practical Encryption Scheme for Online Social Networks}
\label{cha:n}


\section{Cryptographic Goals}
Section~\ref{sec:goals_of_this_thesis} already defined a set of abstract goals for an architecture trying to solve the current privacy issues in OSNs. These goals said that a solution should be user friendly, applicable and immediately ready to use. Besides from these general design goals, it is now possible to define specific cryptographic requirements as well. A well-designed encryption scheme should be able to achieve the following cryptographic goals when publishing a message $m$ to a set of intended recipients $\mathcal{S}$ on an OSN with the help of an encryption scheme:
\begin{itemize}
 \item \textbf{Confidentiality:} The message is protected from disclosure to unauthorised parties, i.e. all entities that are not explicitly in the recipient set $\mathcal{S}$.
 \item \textbf{Authenticity:} The recipients of the message have reasonable assurances of the message's origin.
 \item \textbf{Integrity:} The recipients are assured the message is distributed in its original form as posted by the sender.
 \item \textbf{No redundancy:} The message should be published only once to reach every recipient in the intended recipient set $\mathcal{S}$.
 \item \textbf{Outsider recipient anonymity:} The intended recipients of a broadcasted message should be anonymous to anyone not included in the intended recipient set $\mathcal{S}$. This implies that neither the OSN has to know who the recipients are. (Definition~\ref{def:outsider_anonymity} gives a more formal definition of outsider-anonymity).
 \item \textbf{No key escrow:} Private keys are only disclosed to the owners of the public key. No other entity should be able to have more information on one's secret key in the information theoretic sense.
 \item \textbf{Key validation:} All users of the system should be able to verify the correctness of their keys.
\end{itemize}

\section{Design Decisions}
\subsection{Confidentiality}
Confidentiality can be achieved by applying an encryption scheme before broadcasting a message. Current solutions like Scramble~\cite{art:BeatoKW11} and Persona~\cite{art:BadenBSBS09} rely on rather classic public key infrastructures thereby requiring the OSN user to subscribe to a third party key infrastructure. These key infrastructures are required to authenticate and store the public keys of all security aware users. However, this does not correspond to the general design goals from Section~\ref{sec:goals_of_this_thesis} stating that the proposed solution should be both user friendly and immediately ready to use.

Identity-based encryption (IBE) can be used to achieve both confidentiality and the general design goals from Section~\ref{sec:goals_of_this_thesis}. During the design of our scheme, three IBE schemes were considered as a potential candidate: Boneh and Franklin IBE~\cite{art:BonehF01}, Sakai and Kasahara IBE~\cite{art:SakaiK03} and Gentry IBE~\cite{art:Gentry06}. For a more elaborate discussion on why only these schemes were considered, the reader is referred to Section~\ref{sec:evolution_of_be}.

Table~\ref{tab:ibe_security_comparison} lists the different security properties of all schemes. Merely based on Table~\ref{tab:ibe_security_comparison}, one would prefer the Gentry IBE scheme as it is the only scheme proven secure in the standard model. In the random oracle model, Boneh and Franklin IBE is prefered over Sakai and Kasahara IBE since it relies on the BDH assumption which is more widely accepted than the stronger BDHI assumption.

The execution times of all considered IBE schemes are illustrated in Table~\ref{tab:ibe_performance_comparison}. Experiments were conducted on an Intel Core 2.4 GHz i5 processor with 8 Gb of 1600 MHz DDR3L onboard memory. Pairing computations were implemented using the MIRACL library~\cite{art:Scott03}. The Gentry IBE scheme was first transformed to the asymmetric setting to give a fair basis of comparison. The exact transformed Gentry IBE scheme is depicted in Appendix~\ref{app:gentrys_ibe_scheme}. 

Table~\ref{tab:ibe_performance_comparison} clearly illustrates the price there is to pay for security in the standard model. Therefore, Boneh and Franklin IBE was chosen as the preferred IBE scheme.


\begin{table}
  \centering
  \begin{tabular}{@{}lccr@{}} \toprule
    \multicolumn{3}{r}{Security Proof} \\ \cmidrule(r){2-4}
    IBE Scheme    & IND-ANO-CCA & Standard model & Assumption \\ \midrule
    Boneh and Franklin & \Checkmark & \XSolidBrush  & BDH \\
    Sakai and Kasahare & \Checkmark & \XSolidBrush & BDHI \\
    Gentry & \Checkmark & \Checkmark & q-BDHE \\ \bottomrule
  \end{tabular}
  \caption{Security comparison of considered IBE schemes}
  \label{tab:ibe_security_comparison}
\end{table}

\begin{table}
  \centering
  \begin{tabular}{@{}lrrrr@{}} \toprule
    \multicolumn{4}{r}{Execution time (ms)} \\ \cmidrule(r){2-5}
    IBE Scheme    & IBE.Setup & IBE.Extract & IBE.Encrypt & IBE.Decrypt \\ \midrule
    Boneh and Franklin & 368.10 & 13.84 & 271.90 & 252.82 \\
    Sakai and Kasahare & 1257.72 & 20.49 & 319.83 & 259.17\\
    Gentry & 24.49 & 37.46 & 1136.65 & 911.32 \\ \bottomrule
  \end{tabular}
  \caption{Performance comparison of considered IBE schemes in MIRACL}
  \label{tab:ibe_performance_comparison}
\end{table}

\subsection{Outsider Recipient Anonymity}
The outsider anonymity requirement is imposed on the recipient set since our solution is developed in the context of OSNs where user interaction plays an important role. Therefore, it is useful that members of the intended recipient set $\mathcal{S}$ know each other. For example, suppose that Alice broadcasts an encrypted message intended to Bob and Dylan using a scheme that fully hides the identity of the recipients. This implies that $\id{Bob}, \id{Dylan} \in \mathcal{S}$. As a reaction to Alice's message, Bob wants to write a reply to start a discussion. However, as Bob does not know which other users are allowed to see Alice's message, he can now only encrypt his reply to Alice thereby preventing Dylan from joining the discussion. Nevertheless, this discussion could have been useful to Dylan as well because otherwise Alice would not have included Dylan as a recipient in $\mathcal{S}$ in the first place.

From the outsider-anonymity requirement, it immediately follows that users not necessarily need to be friends to receive each other's messages. In the specific example of Alice, Bob and Dylan, it could be that Bob and Dylan both have Alice as a common friend while no immediate friend connection exists between Bob and Dylan. This should be taken into consideration when determining the identifiers of Bob's and Dylan's profiles, $\id{Bob}$ and $\id{Dylan}$ respectively.

As discussed in Section~\ref{sec:anobe}, broadcast encryption schemes can be made more efficient if the recipient set $\mathcal{S}$ is public. So if user interaction is really that important, why not make the intended recipient set public? Consider the example in which Bob's girlfriend celebrates her birthday in a few weeks. When Bob's girlfriend notices that Bob broadcasted an encrypted message to all her friends without including her as a recipient, she will probably know Bob is up to something. This is just one example of possible many that illustrates the negative impact on security, broadcasting of the recipient set $\mathcal{S}$ can have on real life situations. Depending on the context, information can be deduced about the message without decrypting it to plain text.

\subsection{No redundancy}
From the no redundancy requirement it immediately follows that a broadcast encryption scheme should be used, preferably one that hides the anonymity of recipients in the intended recipient set $\mathcal{S}$ to the outside world. However, apart from the outsider-anonymous broadcast encryption scheme from Fazio and Perera~\cite{art:FazioP12}, no efficient schemes of this kind are described in literature. Since the BE scheme from Fazio and Perera does not fully benefit from the advantages of IBE, the ANOBE scheme from Libert et al.~\cite{art:LibertPQ12} is preferred for further implementation.

Since recipients still have to know who the other recipients within the intended recipient set $\mathcal{S}$ are, the list of \id{}s within the recipient set is concatenated to the plaintext message before encryption.

The scheme from Libert et al. also offers non-repudiation by using signature schemes. Note however, that a trusted authority authorising and publishing the public keys is required for the implementation of signature schemes. Because the general design goals were applicability and user friendliness, no third party PKI can be supported. Therefore, the implemented scheme does not rely on signatures like in~\cite{art:LibertPQ12}.

If the security parameter is chosen to be $\lambda$, the IBE scheme in Algorithm~\ref{alg:full_indent} can only encrypt messages with a maximum length of $l$ bits. This can be seen since in the last step of \texttt{IBE.Encrypt} the message $m$ is encrypted by an XOR operation with the result of a hash function $H_3: \left( 0,1 \right)^l \rightarrow \left( 0,1 \right)^l$. Because asymmetric IBE schemes can only encrypt these fixed length messages, the scheme from Libert et al.~\cite{art:LibertPQ12} is altered such that the ciphertext in the original proposal contains a with IBE encrypted symmetric session key $k$ that is the same for each user in the recipient set $\mathcal{S}$ on a per message basis. The actual plaintext is then encrypted with a symmetric encryption scheme based on a mode of operation to support longer message lengths.

\subsection{Authenticity and Integrity}
Authenticity and integrity can be achieved at the same time by relying on an authenticated encryption scheme (Section~\ref{sec:authenticated_encryption}). The integrity of the message is then as strong as the security guarantees of the authenticated encryption scheme. 

Note however, that the authentication mechanism still relies on the security guarantees of the OSN. Since no third party PKI mechanism is used, there is no trusted party verifying the identity corresponding to a public key. In OSNs this is not an issue if IBE is used with unique profile identifiers as a public key. Consequently, such an IBE scheme ensures that messages encrypted under a public identifier can only be seen by the owner of the corresponding OSN profile. Verifying whoever owns the OSN profile remains the responsibility of the OSN and the judgement of the OSN profile's connections. However, if the authentication mechanism of the OSN is inadequate, anyone could login to a user's profile to impersonate the actual owner of the profile. Therefore, our proposed solution can not be more secure than the authentication mechanism of the OSN.

In more traditional communication schemes, authenticated encryption is done with a symmetric key as agreed during an authenticated key agreement protocol like the Station-to-Station protocol~\cite{art:DiffieOW92}. Authenticity of ciphertexts generated by the authenticated encryption scheme than immediately follows from the usage of the same symmetric session key $k$ as earlier agreed during the protocol. However, since in the proposed solution every OSN user should be able to immediately broadcast confidential messages to other users of the OSN, no key agreement protocols will be used. With the publication of only one broadcast ciphertext, every user in the intended recipient set $\mathcal{S}$ should be immediately able to decrypt it to the original plaintext message $m$. Therefore, there is no real authenticity in the value of the tag $t$ generated by the authenticated encryption scheme because anyone with access to the user's profile could have chosen a random symmetric session key $k$ and have used it as an input to the authenticated encryption scheme. Unless, the only one with access to the user's profile is the actual owner of the profile. Therefore, the authenticity guaranteed by the authenticated encryption scheme boils down to the security of the authentication mechanism as powered by the OSN.


\subsection{No Key Escrow and Key Validation}
% TODO: schrijf iets over de voorwaarden waaraan de ID moet voldoen.
\section{Security Model}


\section{Threat Model}

\section{Proposed Scheme}

\subsection{Scheme}

\subsection{Evaluation}

\section{Conclusion}

%%% Local Variables: 
%%% mode: latex
%%% TeX-master: "thesis"
%%% End: 
