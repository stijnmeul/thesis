\documentclass[master=eelt,masteroption=em]{kulemt}
\setup{title={Practical Identity-Based Encryption for Online Social Networks},
  author={Stijn Meul},
  promotor={Prof.\,dr.\,ir.\ Bart Preneel \and Prof.\,dr.\,ir.\ Vincent Rijmen},
  assessor={Prof.\,dr.\,ir.\,Claudia Diaz\and Prof.\,dr.\,ir.\ Frank Piessens},
  assistant={Filipe Beato}}
% The following \setup may be removed entirely if no filing card is wanted
\setup{filingcard,
  translatedtitle=,
  udc=621.3,
  shortabstract={Here comes a very short abstract, containing no more than 500
    words. \LaTeX\ commands can be used here. Blank lines (or the command
    \texttt{\string\pa r}) are not allowed!
    \endgraf \lipsum[2]}}
% Uncomment the next line for generating the cover page
%\setup{coverpageonly}
% Uncomment the next \setup to generate only the first pages (e.g., if you
% are a Word user.
%\setup{frontpagesonly}

% Choose the main text font (e.g., Latin Modern)
\setup{font=lm}

% If you want to include other LaTeX packages, do it here. 

% Finally the hyperref package is used for pdf files.
% This can be commented out for printed versions.
\usepackage[pdfusetitle,plainpages=false]{hyperref}

\usepackage{kulemtx}
\headstyles{kulemtman}
%%%%%%
% Extra packages
\usepackage{amsthm}
\usepackage{amsfonts}
\usepackage{amsmath}
\usepackage{algorithm}
\usepackage{algpseudocode}
\usepackage{tikz}
\usetikzlibrary{arrows,positioning}

%%%%%%
% Make theorem titles bold
\makeatletter
\def\th@plain{%
  \thm@notefont{}% same as heading font
  \itshape % body font
}
\def\th@definition{%
  \thm@notefont{}% same as heading font
  \normalfont % body font
}
\makeatother

%%%%%%%
% The lipsum package is used to generate random text.
% You never need this in a real master thesis text!
\IfFileExists{lipsum.sty}%
 {\usepackage{lipsum}\setlipsumdefault{11-13}}%
 {\newcommand{\lipsum}[1][11-13]{\par And some text: lipsum ##1.\par}}
%%%%%%%

% Reset theorem numbering for each chapter
\theoremstyle{plain}
\newtheorem{thm}{Theorem}[chapter]

% definition numbers are dependent on the theorem numbers
\theoremstyle{definition}
\newtheorem{defn}[thm]{Definition}

\newcommand{\id}[1]{\ensuremath{\mathtt{id}_{#1}}}

%\includeonly{chap-n}
\begin{document}


\begin{preface}
  I would like to thank everybody who kept me busy the last year,
  especially my promotor and my assistants. I would also like to thank the
  jury for reading the text. My sincere gratitude also goes to my wive and
  the rest of my family.
\end{preface}

\tableofcontents*

\begin{abstract}
  The \texttt{abstract} environment contains a more extensive overview of
  the work. But it should be limited to one page.

  \lipsum[1]
\end{abstract}

% A list of figures and tables is optional
%\listoffigures
%\listoftables
% If you only have a few figures and tables you can use the following instead
\listoffiguresandtables
% The list of symbols is also optional.
% This list must be created manually, e.g., as follows:
\chapter{List of Abbreviations and Symbols}
\section*{Abbreviations}
\begin{flushleft}
  \renewcommand{\arraystretch}{1.1}
  \begin{tabularx}{\textwidth}{@{}p{30mm}X@{}}
    IBE   & Identity-Based Encryption \\
    PKG   & Public Key Generator \\
    IND-CPA  & Indistinguishability under Chosen Plaintext Attack  \\
    IND-CCA & Indistinguishability under Chosen Ciphertext Attack \\
  \end{tabularx}
\end{flushleft}
\section*{Symbols}
\begin{flushleft}
 \renewcommand{\arraystretch}{1.1}
 \begin{tabularx}{\textwidth}{@{}p{30mm}X@{}}
  $G$ & A group $\left( G, * \right)$ \\
  $S_A \left( m \right)$ & Signature of entity $A$ on message $m$ \\
  $e: G_1 \times G_2 \rightarrow G_T$ & Admissible bilinear map \\
  $e \left( P, Q \right)$ & Admissible bilinear map $P \in G_1, Q \in G_2, e \left( P, Q \right) \in G_T$ \\  
  $\mathcal{A} \left( a, b \right)$ & An algorithm $\mathcal{A}$ with parameters $a$ and $b$ \\
  $\left< a, b, c \right> \leftarrow \mathcal{A}( d, e )$ & Algorithm $\mathcal{A}$ with parameters $d$ and $e$ returns the collection of values $a, b, c$ \\
 \end{tabularx}
\end{flushleft}

% Now comes the main text
\mainmatter

\chapter{Introduction}
\label{cha:intro}
The online social network (OSN) is the most impactful internet trend at the dawn of the 21st century. Words like tweeting, sharing, liking, trending and tagging have found common acceptance in the vocabulary of current internet users, while services like Facebook, Google+, LinkedIn and Twitter have become part of everyday life. OSNs offer millions of users an efficient and reliable channel to distribute and share information. At the same time, OSNs store large amounts of data which prompts several privacy concerns. In particular, it is possible to infer a considerable amount of sensitive information from the shared and stored content. Currently, users are allowed to configure ''privacy preferences'' in order to limit and select which users or groups can access the shared content. These preferences are generally too coarse-grained and difficult to configure~\cite{bonneau2010privacy}. Another problem is that these preferences do not exclude the provider along with the dangers of data leaks~\cite{fischetti11hacker} nor external governments~\cite{prism}.


%In May 2013, 72\% of all internet users were active on a social network~\cite{site:Jones13}. Currently, Facebook has 1.23 Billion monthly active users which corresponds to 17\% of the global population~\cite{site:Bullas14,site:worldometers}. Furthermore, the average Facebook user spends 15 hours and 33 minutes online per month~\cite{site:StatisticBrain}. These numbers show that social networks no longer represent the latest craze of an internet bubble. Conversely, OSNs are deeply rooted in our daily habits.

\section{Problem Statement}
\label{sec:problem_statement}
All these worrisome issues motivate the need for effective techniques to properly protect user's privacy in OSNs. Several solutions have been proposed and advocated to use cryptographic mechanisms in order to address the privacy issues, either by an add-on atop of existing OSNs~\cite{art:BadenBSBS09,art:BeatoKW11,art:GuhaSTF08,art:LuoXH09}, or by complete new privacy-friendly architectures~ \cite{art:CristofaroSTW11}, mainly decentralised~\cite{art:CutilloMO11,NYT2010.Diaspora}. In general, those solutions suffer from user adoption and key management issues as users are required to register and then share, certify and store public keys~\cite{art:BalseBADG14}. Completely new architectures represent a difficult step for users as the trade off of moving away from the commonly used social ecosystem compared with the risk of losing interactions is high. Arguably, current centralised OSNs are here to stay and will be continue to be actively used by millions of people. In light of recent events, such as Edward Snowden's whistle-blowing on US surveillance programs~\cite{prism}, OSN providers have all interest to maintain their users and a privacy-friendly image. However, not all OSNs are willing to cooperate in a more private infrastructure since targeted advertising is the most important part of their revenues.

\section{Existing Solutions}
\label{sec:existing_solutions}
Several existing solutions have been proposed in literature, all trying to solve most of the earlier mentioned issues in OSNs.

\paragraph{flyByNight~\cite{art:LucasB09}} is a Facebook application that protects user data by storing it in encrypted form on Facebook. It relies on Facebook servers for its key management and is thus not secure against active attacks by Facebook itself.

\paragraph{NOYB (None Of Your Business)~\cite{art:GuhaSTF08}} replaces details of a user with details from other random users thereby making this process only reversible by friends. However, the proposed solution does not apply to user messages or status updates that are the most frequently used features in the OSNs considered in this thesis.

\paragraph{FaceCloak~\cite{art:LuoXH09}} stores published Facebook data on external servers in encrypted form and replaces the data on Facebook with random text from Wikipedia. This could be a useful mechanism to prevent OSNs from blocking security aware users because they are scared to see their advertising revenues shrink. However, this approach has the disadvantage that other users could take this data as genuine user content which may lead to social issues. Furthermore, FaceCloaks architecture leads to an inefficient key distribution system.

\paragraph{Persona~\cite{art:BadenBSBS09}} is a scheme that can be used as a Firefox extension to let users of an OSN determine their own privacy by supporting the ability to encrypt messages to a group of earlier defined friends based on \textit{attribute-based encryption} (ABE)~\cite{art:SahaiW04}. The scheme supports a wide range of meaningful use cases. For instance, sending messages to all friends that are related to a certain attribute or even encrypting messages to friends of friends. However, the major drawback of this system is that, for every new friend  a public key is exchanged before he is able to interact in the privacy preserving architecture consequently requiring an infrastructure for broadcasting and storing public keys. Furthermore, to support the encryption of messages to friends of friends, user defined groups should be made available publicly thereby making the public key distribution system even more complicated. Finally the proposed ABE encryption scheme is 100 to 1000 times slower than a standard RSA operation~\cite{art:BadenBSBS09}.

\paragraph{Scramble~\cite{art:BeatoKW11}} is a Firefox extension that allows users to define groups of friends that are given access to stored content on OSNs. The tool uses public key encryption based on OpenPGP~\cite{rfc4880} to broadcast encrypted messages on any platform. Furthermore Scramble provides the implementation of a tiny link server such that OSN policies not allowing to post encrypted data are bypassed. However, as indicated by usability studies~\cite{art:WhittenT99} OpenPGP has a higher usage threshold because an average user does not manage to understand OpenPGP properly. Additionally, Scramble has to rely on the security decisions of the web of thrust. It therefore inherits the unpleasant property of OpenPGP that the user can not be sure that the used PGP key actually belongs to the intended Facebook profile.

The most unattractive property of all the above applications is that they have to rely on a rather complex infrastructure. Persona has to support an extended public key distribution system and Scramble relies on the leap-of-faith OpenPGP web of trust. All proposed solutions require users with no cryptographic background on asymmetric cryptography to make responsible decisions concerning the management of their keys. Furthermore, maintaining such complex key infrastructures becomes more and more complex as more users subscribe.

\section{Goals of this Thesis}
\label{sec:goals_of_this_thesis}
The goal of this thesis is to develop an architecture that solves the issues discussed in Section~\ref{sec:problem_statement} thereby taking the challenges and pitfalls from earlier solutions in Section~\ref{sec:existing_solutions} into account. Specifically, the architecture should present the following properties:
\begin{itemize}
 \item \textbf{User friendly:} The average OSN user should be able to use the resulting architecture, i.e. a user with no knowledge on cryptographic primitives.
 \item \textbf{Applicable:} The original OSN environment should not be altered since some OSN providers are probably not willing to support a more confidential architecture because it could possibly hurt their business model.
 \item \textbf{Immediately ready to use:} No additional registration or subscription to third party key architectures should be required to enable usage of the system. As soon as a user subscribes to the OSN provider he should be able to start receiving confidential messages.
\end{itemize}


\section{Structure of this Thesis}

%%% Local Variables: 
%%% mode: latex
%%% TeX-master: "thesis"
%%% End: 

\chapter{Preliminaries}
\label{cha:1}
This chapter covers briefly the mathematical knowledge required to understand cryptographic algorithms presented later in this text. Although understanding all mathematical details of this chapter can be quite a struggle, the math serves as a fundament of a challenging world containing exciting cryptographic concepts like identity-based encryption.

First, the notion of negligible functions will be introduced followed by an overview of algebraic structures and their properties. Then, a number of theoretic assumptions fundamental for cryptographic security are presented. By exploring these variants of the Diffie-Hellman assumption, the introduction of gap groups and bilinear maps follows naturally. Finally, hash functions are defined as well as their relation to the random oracle assumption.

Note that this chapter only scratches the surface of cryptographic fundamentals required to understand the remainder of the thesis. Definitions and theorems are always provided without proof. For a more in depth discussion about algebraic topics in this chapter, the reader is referred to~\cite{book:handbook_of_applied_cryptography} and~\cite{book:survey_of_modern_algebra}. More information on elliptic curves, Diffie-Hellman assumptions and pairing based cryptography can be found in~\cite{thesis:Maas04}.

If the reader feels he has sufficient background of the concepts covered in this chapter, the chapter can be skipped without loss of comprehension.

\section{Negligible Function}

In practice no modern cryptographic algorithm achieves perfect secrecy\footnote{Note that the one-time pad is not taken into account. Although it is the only proven information secure cryptographic algorithm, it is seldom used in practical cryptographic systems.}, i.e. with unbounded computational power all practical cryptographic algorithms can be broken. Therefore a more pragmatic definition of security is always considered, namely security against adversaries that are computationally bound to their finite resources. In this pragmatic view of security an algorithm is considered secure only if the probability of success is smaller than the reciprocal of any polynomial function. The negligible function can be used to exactly describe this notion in a formal way.

\begin{defn}
\label{def:negligible_function}
A \textbf{negligible function} in $k$ is a function $\mu \left( k \right): \mathbb{N} \rightarrow \mathbb{R}$ if for every polynomial $p \left( . \right)$ there exists an $N$ such that for all $k > N$~\cite{book:Goldreich97}
 \begin{equation*}
  \mu \left( k \right) < \frac{1}{p\left( k \right)} 
 \end{equation*}
\end{defn}

The negligible function will be used later on in this chapter to formally describe computationally infeasible problems. In such a context $k$ often represents the security parameter. The larger $k$ will be chosen, the smaller $\mu \left( k \right)$ will be.

\section{Abstract Algebra}
Abstract algebra is a field of mathematics that studies algebraic structures such as groups, rings and vector spaces. These algebraic structures define a collection of requirements on mathematical sets such as e.g., the natural numbers $\mathbb{N}$ or matrices of dimension 2 x 2 $\mathbb{R}^{2 x 2}$. If these requirements hold, abstract properties can be derived. Once a mathematical set is then categorised as the correct algebraic structure, properties derived for the algebraic structure will hold for the set as a whole.

In the light of our further discussion, especially additive and multiplicative groups prove to be essential concepts. However, algebraic groups come with a specific vocabulary such as binary operation, group order and cyclic group that are defined in this section as well.

\begin{defn}[Binary operation]
 A \textit{binary operation} * on a set $S$ is a mapping $S \times S \rightarrow S$. That is, a binary operation is a rule which assigns to each ordered pair of elements $a$ and $b$ from $S$ a uniquely defined third element $c = a*b$ in the same set $S$.~\cite{book:handbook_of_applied_cryptography,book:survey_of_modern_algebra}
\end{defn}

\begin{defn}[Group]
\label{def:group}
 A \textit{group} $\left( G, * \right)$ consists of a set $G$ with a binary operation $*$ on $G$ satisfying the following three axioms:
 \begin{enumerate}
  \item \textit{Associativity} $\forall a, b, c \in G: a*(b*c) = (a*b)*c$
  \item \textit{Identity element} $\forall a \in G, \exists e \in G: a*e = e*a = a $ where $e$ denotes the \textit{identity element} of $G$
  \item \textit{Inverse element} $\forall a \in G, \exists a^{-1}: a*a^{-1} = a^{-1}*a = 1$ where $a^{-1}$ denotes the \textit{inverse element} of $a$
  \newcounter{enumTemp}
  \setcounter{enumTemp}{\theenumi}
 \end{enumerate}

\end{defn}

\begin{defn}[Commutative group]
 A group $\left( G, * \right)$ is called a \textit{commutative group} or an \textit{abelian group} if in addition to the properties in Definition~\ref{def:group}, also commutativity holds.
 \begin{enumerate}
  \setcounter{enumi}{\theenumTemp}
  \item \textbf{Commutativity} $\forall a, b \in G: a*b = b*a$
 \end{enumerate}

\end{defn}

Depending on the group operation~$*$, $\left( G, * \right)$ is called either a \textit{multiplicative group} or an \textit{additive group}. In Definition~\ref{def:group} the multiplicative notation is used. For an additive group  the inverse of $a$ is often denoted $-a$~\cite{book:handbook_of_applied_cryptography}. 

A group $\left( G, * \right)$ is often denoted by the more concise symbol $G$ although groups are always defined with respect to a binary group operation $*$. Despite of a more concise notation, any group $G$ still obeys all axioms from Definition~\ref{def:group} with respect to an implicitly known group operation $*$.

A perfect example of a commutative group is the set of integers with the addition operation $\left( \mathbb{Z}, + \right)$ since the addition is both associative and commutative in $\mathbb{Z}$. Furthermore, the identity element $e = 0$ and the inverse element $\forall a \in \mathbb{Z}$ is $-a \in \mathbb{Z}$. Note that the set of natural numbers with the addition operation $\left( \mathbb{N}, + \right)$ is not a commutative group as not every element of $\mathbb{N}$ has an inverse element.

\begin{defn}[Cyclic group]
\label{def:cyclic_group}
 A group $G$ is \textit{cyclic} if and only if $\forall b \in G, \exists g \in G,\exists n \in \mathbb{Z}: g^n = b$. Such an element $g$ is called a \textbf{generator} of $\mathbb{G}$.
\end{defn}

Definition~\ref{def:cyclic_group} implies that in a cyclic group every element can be written as a power of one of the group's generators.

\begin{defn}[Finite group]
\label{def:finite_group}
 A group $G$ is \textit{finite} if the number of elements in $G$ denoted $|G|$ is finite. The number of elements $|G|$ in a finite group is called the \textit{group order}.
\end{defn}

The set $\mathbb{Z}_n$ denotes the set of integers modulo $n$. The set $\mathbb{Z}_5$ with the addition operation is a cyclic finite group of order 5. The set $\mathbb{Z}_5 \backslash \{0\}$ with the multiplication operation, often denoted $\mathbb{Z}^{*}_5$, is a cyclic finite group of order 4 where the neutral element $e=1$. Two is an example of a generator in $\mathbb{Z}^{*}_5$ since every element in $\mathbb{Z}^{*}_5$ can be written as $\{ 2^n | n \in \mathbb{Z} \}$.

\begin{defn}[Order of an element]
\label{def:order_of_an_element}
Let $G$ be a group. The \textit{order of an element} $a \in G$ is defined as the least positive integer $t$ such that $a^t = e$. If there exists no such $t$, $t$ is defined as~$\infty$.
\end{defn}

\begin{thm}
\label{the:group_modulo_a_prime}
If the order of a group $G$ equals a prime $p$, the group is cyclic and commutative.
\end{thm}

\begin{defn}[Subgroup]
\label{def:subgroup}
 Given a group $\left( G, * \right)$, any $H$ that is a non-empty subset $H \subseteq G$ and satisfies the axioms of a group with respect to the group operation $*$ in $H$, is a \textit{subgroup of $G$}.
\end{defn}

\begin{defn}[Ring]
\label{def:ring}
  \setcounter{enumTemp}{\theenumi}
 A \textit{ring} $\left( R, +, * \right)$ consists of a set $R$ with two binary operations $+$ and $*$ on $R$ satisfying the following axioms:
 \begin{enumerate}
  \item $\left( R, + \right)$ is an abelian group with identity denoted $e$
  \item \textit{Associativity} $\forall a, b, c \in R: a*(b*c) = (a*b)*c$
  \item \textit{Multiplicative identity element} $\forall a \in R, \exists 1 \in R: a*1 = 1*a = a $ where $1$ denotes the \textit{multiplicative identity element} of $R$
  \item \textit{Left distributivity} $\forall a, b, c \in R: a*\left( b + c \right) = \left( a * b \right) + \left( a * c \right)$
  \item \textit{Right distributivity} $\forall a, b, c \in R: \left( b + c \right) * a = \left( b * a \right) + \left( c * a \right)$
   \setcounter{enumTemp}{\theenumi}
 \end{enumerate}
\end{defn}

\begin{defn}[Commutative ring]
 \label{def:commutative_ring}
 A ring $\left( R, +, * \right)$ is called a \textit{commutative ring} or an \textit{abelian ring} if in addition to the properties in Definition~\ref{def:ring}, also commutativity holds.
 \begin{enumerate}
  \setcounter{enumi}{\theenumTemp}
  \item \textbf{Commutativity} $\forall a, b \in R: a*b = b*a$
  \setcounter{enumTemp}{\theenumi}
 \end{enumerate}

\end{defn}

\begin{defn}[Field]
\label{def:field}
 A commutative ring $\left( R, +, * \right)$is called a \textit{field} if in addition to the properties in Definition~\ref{def:commutative_ring} and Definition~\ref{def:ring} all elements of $R$ have a multiplicative inverse.
 \begin{enumerate}
  \setcounter{enumi}{\theenumTemp}
   \item \textit{Multiplicative inverse} $\forall a \in R, \exists a^{-1}: a*a^{-1} = a^{-1}*a = 1$ where $a^{-1}$ denotes the \textit{inverse element} of $a$
 \end{enumerate}
\end{defn}

\begin{defn}[Finite field]
\label{def:finite_field}
 A \textit{finite field} or a \textit{Galois Field} is a field $F$ with a finite number of elements. The number of elements $|F|$ of a finite field $F$ is called its \textit{order}.
\end{defn}

\begin{defn}[Ring homomorphism]
\label{def:ring_homomorphism}
 Given rings $R$ and $S$, a \textit{ring homomorphism} is a function $f: R \rightarrow S$ such that the following axioms hold:
 \begin{enumerate}
  \item $\forall a, b \in R: f \left( a + b \right) = f \left( a \right) + f \left( b \right)$
  \item $\forall a, b \in R: f \left( ab \right) = f \left( a \right) f \left( b \right)$
  \item $f \left( e_R \right) = f \left( e_S \right)$ where $e_S$ and $e_R$ denote the identity element of respectively $S$ and $R$
 \end{enumerate}
\end{defn}

\begin{defn}[Bijective function]
\label{def:bijective_function} 
 Any function $f: R \rightarrow S$ is bijective if it satisfies the following axioms
 \begin{enumerate}
  \item \textit{Injective} Each element in $S$ is the image of at most one element in $R$. Hence, $\forall a_1, a_2 \in R$ if $\left( a_1 \right) = f \left( a_2 \right)$ then $a_1 = a_2$ follows naturally.
  \item \textit{Surjective} Each $s \in S$ is the image of at least one $r \in R$.
 \end{enumerate}
\end{defn}

\begin{defn}[Ring isomorphism]
\label{def:ring_isomorphism}
 A ring isomorphism is a bijective homomorphism.
\end{defn}

Informally speaking, a ring isomorphism $f: R \rightarrow S$ is a mapping between rings that are structurally the same such that any element of $R$ has exactly one image in $S$.

Note that $\left( \mathbb{Z}_n, +, \cdot \right)$ is a finite field if and only if $n$ is a prime number. Furthermore, if $F$ is a finite field, then $F$ contains $p^m$ elements for some prime $p$ and integer $m \geq 1$. For every prime power order $p^m$, there is a unique finite field of order $p^m$. This field is denoted by $\mathbb{F}_{p^m}$ or $GF \left( p^m \right)$. The finite field $\mathbb{F}_{p^m}$ is unique up to an isomorphism. 


\section{Number Theoretic Assumptions}
\label{sec:number_theoretic_assumptions}
This section presents a collection of number theoretic assumptions. The security of our future constructions falls or stands on these assumptions. If one of these assumptions would prove to be invalid, not only this thesis would be superfluous, society would no longer be protected by widely adopted cryptographic protocols like RSA or ElGamal encryption~\cite{art:Boneh98,book:handbook_of_applied_cryptography}.

In the definitions that follow $\left< G, n, g \right> \leftarrow \mathcal{G} \left( 1^k \right)$ is defined as the setup algorithm that generates a group $G$ of order $n$ and a generator $g \in G$ on input of the security parameter $k$.

\begin{defn}[DL]
\label{def:dl}
The \textit{discrete logarithm problem} is defined as follows. Given a finite cyclic group $G$ of order $n$, a generator $g \in G$ and an element $a \in G$, find the integer $x, 0 \leq x \leq n-1$ such that $g^x = a$.

The \textit{discrete logarithm assumption} holds if for any algorithm $\mathcal{A} \left( g, g^x \right)$ trying to solve the DL problem there exists a negligible function $\mu \left( k \right)$ such that 
 \begin{equation*}
  \textrm{Pr} \left[ \mathcal{A} \left( g, g^x \right) = a \mid \left< G, n, g \right> \leftarrow \mathcal{G} \left( 1^{k} \right)\right] \leq \mu \left( k \right)
 \end{equation*}
 where the probability is over the random choice of $n, g$ in $G$ according to the distribution induced by $\mathcal{G} \left( 1^k \right)$, the random choice of $a$ in $G$ and the random bits of the algorithm $\mathcal{A}$.
\end{defn}


\begin{defn}[CDH]
\label{def:cdh}
The \textit{Computational Diffie-Hellman problem} is defined as follows. Given a finite cyclic group $G$ of order $n$, a generator $g \in G$ and $g^a, g^b$ with uniformly chosen random independent elements $a, b \in \{ 1, \ldots, | G |\}$ , find the value $g^{ab}$.


The \textit{Computational Diffie-Hellman assumption} holds if for any algorithm $\mathcal{A} \left( g, g^a, g^b \right)$ trying to solve the CDH problem there exists a negligible function $\mu \left( k \right)$ such that 
 \begin{equation*}
  \textrm{Pr} \left[ \mathcal{A} \left( g, g^a, g^b \right) = g^{ab} \mid \left< G, n, g \right> \leftarrow \mathcal{G} \left( 1^{k} \right)\right] \leq \mu \left( k \right)
 \end{equation*}
 where the probability is over the random choice of $n, g$ in $G$ according to the distribution induced by $\mathcal{G} \left( 1^k \right)$, the random choice of $a, b$ in $\{ 1, \ldots, | G |\}$ and the random bits of the algorithm $\mathcal{A}$.
\end{defn}

\begin{defn}[DDH]
\label{def:ddh}
The \textit{Decisional Diffie-Hellman problem} is defined as follows. Given a finite cyclic group $G$ of order $n$, a generator $g \in G$ and $g^a, g^b, g^{ab}, g^c$ with uniformly chosen random independent elements $a, b, c \in \{ 1, \ldots, | G |\}$, distinguish $\left< g, g^a, g^b, g^{ab} \right>$ from $\left< g, g^a, g^b, g^c \right>$.

Define $\mathcal{A} \left( x \right)$ as an algorithm returning \texttt{true} if $x = \left< g, g^a, g^b, g^{ab} \right>$ and \texttt{false} if $x = \left< g, g^a, g^b, g^c \right>$ for $c \neq ab$. The \textit{Decisional Diffie-Hellman assumption} holds if for any such algorithm $\mathcal{A} \left( x \right)$ there exists a negligible function $\mu \left( k \right)$ such that 
 \begin{equation*}
  \lvert \textrm{Pr} \left[ \mathcal{A} \left( \left< g, g^a, g^b, g^{ab} \right> \right) = \texttt{true} \right] - \textrm{Pr} \left[ \mathcal{A} \left( \left< g, g^a, g^b, g^{c} \right> \right) = \texttt{true} \right] \rvert \leq \mu \left( k \right)
 \end{equation*}
 where the probability is over the random choice of $n, g$ in $G$ according to the distribution induced by $\mathcal{G} \left( 1^k \right)$, the random choice of $a, b, c$ in $\{ 1, \ldots, | G | \} $ and the random bits of the algorithm $\mathcal{A}$.
\end{defn}


Definition~\ref{def:ddh} states that $\left< g, g^a, g^b, g^{ab} \right>$ and $\left< g, g^a, g^b, g^{c} \right>$ are \textit{computationally indistinguishable}. It means that no efficient algorithm exists that can distinguish both arguments with non-negligible probability. The concept of computational indistinguishable arguments bears close resemblance to statistically indistinguishable ensembles. The reader is referred to~\cite{art:Goldwasser84} and~\cite{art:Goldwasser89} for a more in depth discussion of the topic. The intuitive interpretation of Definition~\ref{def:ddh} is that $g^{ab}$ looks like any other random element in $G$.

Someone with the ability to calculate discrete logarithms could trivially solve the CDH problem. That is, if $a$ and $b$ can be derived only from $\left< g^a, g^b \right>$, it becomes easy to calculate $g^{ab}$. Therefore, a group structure where the CDH assumption holds, immediately implies a group where the DL assumption is valid as well. There is no mathematical proof that supports the inverse relation. Thus, a group where the DL problem is hard not necessarily implies the CDH problem. For specific group structures~\cite{art:MaurerW98} and~\cite{art:MaurerW99} show that CDH immediately follows from the DL assumption, however, their proof can not be generalised to just any group.

There exists a similar relation between the CDH and the DDH problem. If a powerful algorithm could solve CDH, i.e. derive $g^{ab}$ from $\left< g, g^a, g^b \right>$ alone, it would become trivial to distinguish $\left< g, g^a, g^b, g^{ab} \right>$ from $\left< g, g^a, g^b, g^c \right>$. Again, an inverse relation can not be proven. As a matter of fact, concrete examples of groups exist where CDH is hard although DDH is not.

Therefore, the relation between DL, CDH and DDH is often written as follows
\begin{equation*}
 DDH \Rightarrow CDH \Rightarrow DL
\end{equation*}
The $\Rightarrow$ notation is then translated into "immediately implies". In a group where DDH is hard both CDH and DL will be hard. On the contrary, there exist group structures where the CDH and the DL assumption hold while DDH can be found easily. Such groups are called \textit{Gap Diffie-Hellman Groups}.

\begin{defn}[GDH]
\label{def:gdh}
The \textit{Gap Diffie-Hellman problem} is defined as follows. Solve the CDH problem with the help of a DDH oracle. Given a finite cyclic group $G$ of order $n$, a generator $g \in G$ and $g^a, g^b$ with uniformly chosen random independent elements $a, b \in \{ 1, \ldots, | G |\}$ , find the value $g^{ab}$ with the help of a DDH oracle $\mathcal{DDH} \left( g, g^a, g^b, z \right)$. Where the DDH oracle $\mathcal{DDH} \left( g, g^a, g^b, z \right)$ is defined to return \texttt{true} if $z = g^{ab}$ and \texttt{false} if $z \neq g^{ab}$.

The \textit{Gap Diffie-Hellman assumption} holds if for any algorithm $\mathcal{A} \left( g, g^a, g^b \right)$ trying to solve the CDH problem with the help of a DDH oracle $\mathcal{DDH} \left( g, g^a, g^b, z \right)$ there exists a negligible function $\mu \left( k \right)$ such that 
 \begin{equation*}
  \textrm{Pr} \left[ \mathcal{A} \left( g, g^a, g^b \right) = g^{ab} \mid \left< G, n, g \right> \leftarrow \mathcal{G} \left( 1^{k} \right)\right] \leq \mu \left( k \right)
 \end{equation*}
 where the probability is over the random choice of $n, g$ in $G$ according to the distribution induced by $\mathcal{G} \left( 1^k \right)$, the random choice of $a, b$ in $\{ 1, \ldots, | G |\}$ and the random bits of the algorithm $\mathcal{A}$.
\end{defn}

As discussed in the next Section~\ref{sec:bilinear_map} bilinear pairings are an example of a practical usable DDH oracle~\cite{art:JouxN03}.

\section{Bilinear Maps}
\label{sec:bilinear_map}

\subsection{Definition}

\begin{defn}[Admissible bilinear map]
\label{def:admissibile_bilinear_map}
 Let $G_1, G_2$ and $G_T$ be three groups of order $q$ for some large prime $q$. An \textit{admissible bilinear map} $e: G_1 \times G_2 \rightarrow G_T$ is defined as a map from the gap groups $G_1$ and $G_2$ to the target group $G_T$ that satisfies the following properties:
 \begin{enumerate}
  \item \textit{Bilinearity} $\forall a, b \in \mathbb{Z}, \forall P \in G_1, \forall Q \in G_2: e \left( aP, bQ \right) = e \left( P, Q \right)^{ab}$
  \item \textit{Non-degeneracy} If $P$ is a generator of $G_1$ and $Q$ is a generator of $G_2$, $e \left( P, Q \right)$ is a generator of $G_T$
  \item \textit{Computability} There is an efficient algorithm to compute $e \left( P, Q \right)$ for all $P \in G_1$ and $Q \in G_2$
 \end{enumerate}

\end{defn}

In literature, authors distinguish two types of admissible bilinear maps. A \textit{symmetric bilinear map} is an admissible bilinear map where the gap groups are the same, i.e. $G_1 = G_2$. Definition~\ref{def:admissibile_bilinear_map} describes the more general \textit{asymmetric bilinear map} where $G_1 \neq G_2$. Schemes relying on symmetric bilinear maps are easier to construct information theoretic security proofs although asymmetric bilinear maps are more efficient and suitable for implementation thanks to their flexible embedding degree~\cite{art:BonehF01,art:ZhangW13}.

In practice, bilinear maps are constructed using pairings. The most popular pairings implementing admissible bilinear maps are the Weil pairing~\cite{art:BonehF01} and the Tate pairing~\cite{art:FreyMR99}. Both the Tate and the Weil pairing rely on abelian varieties for their implementation. $G_1$ is mostly an additive elliptic curve group, $G_2$ a multiplicative elliptic curve group while $G_T$ is a finite field. For instance, the asymmetric Weil pairing is often implemented with a cyclic subgroup of $E\left( \mathbb{F}_p \right)$ of order $q$ for $G_2$ and a different cyclic subgroup of $E \left( \mathbb{F}_{p^6} \right)$ of the same order $q$ for $G_1$ where $E\left( \mathbb{F}_{p^6} \right)$ denotes the group of points on an elliptic curve $E$ over the finite field $\mathbb{F}_{p^6}$. The interested reader is referred to~\cite{thesis:Maas04} for more information concerning elliptic curves and their use in pairing based cryptography. Details on Elliptic Curve Cryptography fall out of the scope of this thesis as it suffices to make abstraction of these concepts for the remainder of the text.

Research~\cite{art:BarbulescuGJT14,art:Joux13,art:AdjMOR13} has recently shown that the discrete logarithm problem is easier in the symmetric setting because symmetric pairings rely on more structured supersingular (hyper)elliptic curves. Therefore, care should be taken when using symmetric pairings~\cite{art:ZhangW13}.

\subsection{Bilinear Diffie-Hellman Assumption}
It is not a coincidence that $G_1$ and $G_2$ are called gap groups. A bilinear map allows to solve the Decisional Diffie-Hellman problem in $G_1$ and $G_2$. The DDH problem in $G_1$ consists of distinguishing $\left< P, aP, bP, abP \right>$ from $\left< P, aP, bP, cP \right>$ where $P \in G_1$, $P$ is a generator of $G_1$ and $a, b, c$ randomly chosen in $\{1, \ldots, \vert G_1 \vert \}$. Given a symmetric bilinear map $e: G_1 \times G_1 \rightarrow G_T$ a solution to this problem can be found by relying on the bilinearity of the pairing as follows:

\begin{equation*}
 e \left( aP, bP \right) = e \left( P, P \right)^{ab} \stackrel{?}{=} e \left( P, cP \right) = e \left( P, P\right)^c
\end{equation*}
Such that the second equality will hold only if $ab = c$. A similar statement can be made concerning $G_2$ with the help of the map $e: G_2 \times G_2 \rightarrow G_T$. Consequently $G_1$ and $G_2$ are both GDH groups. From Section~\ref{sec:number_theoretic_assumptions} it follows that CDH can still be hard in GDH groups because DDH is a stronger assumption~\cite{art:BonehF01}.

Since DDH in the Gap groups $G_1$ and $G_2$ is easy, DDH can not serve as a basis for crypto systems in these groups. Therefore, an alternative to the CDH problem is defined called the Bilinear Diffie-Hellman problem.

In the definition that follows $\mathcal{G} \left( 1^k \right)$ is defined to be a BDH parameter generator as in~\cite{art:BonehF01}, i.e. $\mathcal{G}$ takes as input a security parameter $k$, $\mathcal{G}$ runs in polynomial time in $k$ and $\mathcal{G}$ outputs a prime number $q$, the description of two groups $G_1, G_2$ of order $q$ and the description of an admissible bilinear map $e: G_1 \times G_2 \rightarrow G_T$ .
\begin{defn}[BDH]
\label{def:bdh}
The \textit{Bilinear Diffie-Hellman problem} is defined as follows. Given any admissible bilinear pairing $e: G_1 \times G_2 \rightarrow G_T$ with random $P, aP, bP \in G_1$ and random $Q, aQ, bQ \in G_2$ with uniformly chosen random independent elements $a, b, c \in \{ 1, \ldots, | G |\}$, find $e \left( P, Q \right)^{abc}$

The \textit{Bilinear Diffie-Hellman assumption} holds if for any algorithm \\ $\mathcal{A} \left( P, aP, bP, Q, aQ, bQ \right)$ trying to solve the BDH problem there exists a negligible function $\mu \left( k \right)$ such that 
 \begin{equation*}
  \textrm{Pr} \left[ \mathcal{A} \left( P, aP, bP, Q, aQ, bQ \right) = e \left( P, Q \right)^{abc} \mid \left< q, G_1, G_2, e \right> \leftarrow \mathcal{G} \left( 1^{k} \right)\right] \leq \mu \left( k \right)
 \end{equation*}
 where the probability is over the random choice of $q, G_1, G_2, e$ according to the distribution induced by $\mathcal{G} \left( 1^k \right)$, the random choice of $a, b$ in $\{ 1, \ldots, | G |\}$ and the random bits of the algorithm $\mathcal{A}$.
\end{defn}

\section{Hash Functions}
\label{sec:hash_functions}
\subsection{Definition}
A \textit{hash function} is a computationally efficient deterministic function mapping binary strings of arbitrary length to binary strings of some fixed length, called \textit{hash-values}.

Cryptographic hash functions have the following desirable properties:
\begin{itemize}
 \item \textit{Computability:} Given a binary string $m$, the hash value $h$ can be calculated efficiently $h = \texttt{hash} \left( m \right)$
 \item \textit{Pre-image resistance:} Given a hash value $h$, it is infeasible to calculate a corresponding binary string $m$ such that $h = \texttt{hash} \left( m \right)$
 \item \textit{Second pre-image resistance:} Given a binary string $m_1$, it is hard to find a different binary string $m_2$ such that $\texttt{hash} \left( m_1 \right) = \texttt{hash} \left( m_2 \right)$
 \item \textit{Strong collision resistance:} Given a \texttt{hash} function \texttt{hash(.)}, it is hard to find two different binary strings $m_1$ and $m_2$ such that $\texttt{hash} \left( m_1 \right) = \texttt{hash} \left( m_2 \right)$
\end{itemize}

Hash functions are useful in a wide plethora of practical applications. Hash functions serve as one way functions in password databases to relax sensitivity of the stored content. Furthermore hash functions are a valuable tool for data authentication and integrity checking. Another use of hash functions is in protocols involving a priori commitments. If the reader is new to the concept of hash functions, he is referred to~\cite{book:handbook_of_applied_cryptography} for an in depth discussion on the topic.

\subsection{Random Oracles}
A \textit{random oracle} is a theoretical black box that returns for each unique query a uniformly random chosen result from its output domain. A random oracle is deterministic, i.e. given a particular input it will always produce the same output.

In a perfect world hash functions can be considered random oracles. That is, if hash functions were perfect, they would behave as random oracles. Therefore, hash functions are often considered random oracles in security proofs. Such security proofs are said to be \textit{proven secure in the random oracle model}. Proofs in the random oracle model first show that an algorithm is secure if a theoretical random oracle would be used. A next step of these security proofs is replacing the random oracle accesses by the computation of an appropriately chosen (hash) function $h$~\cite{art:BellareR93}. Algorithms that do not require such a construction in their security proof are said to be \textit{proven secure in the standard model}.

Although theoretical definitions of random oracles and hash functions are quite similar, some practical implementations of hash functions do not behave like random oracles at all. Canetti at al. show that there exist signature and encryption schemes that are secure in the Random Oracle Model, although any implementation of the random oracle results in insecure schemes~\cite{art:CanettiGH04}. Coron et al. counter these findings with indifferentiability, i.e. if a hash function is indifferentiable from a random oracle the random oracle can be replaced by the hash function while maintaining a valid security proof~\cite{art:CoronDMP05}. Although research results from Coron et al. are debated in~\cite{art:FleischmannGL10} and~\cite{art:RistenpartSS11}, it is a common belief that proofs in the random oracle model provide some evidence that a system is secure. As a matter of fact, indifferentiability from random oracles certainly contributed to the victory of Keccak in the NIST hash function competition for a new SHA-3 hashing standard as all final round hashing algorithms supported this property~\cite{art:BartheGHOB13}.

\section{Conclusion}
The first part of this chapter introduced the concepts of a negligible function as well as algebraic structures such as groups and finite fields. These basic notions were used further on to define number theoretic hard problems that serve as a basis for security. Starting from the discrete logarithm assumption, several variants of the Diffie-Hellman problem were introduced eventually leading to the Gap Diffie-Hellman assumption. Notion of the Gap Diffie-Hellman assumption allowed to uncover gap groups and their use in admissible bilinear maps. The Bilinear Diffie-Hellman assumption was defined as a computationally infeasible problem for the construction of cryptographic protocols relying on bilinear maps. Finally, this chapter concluded with differences between security under random oracle assumptions and security in the standard model. 

Now the reader has knowledge of the mathematic fundaments, more advanced cryptographic constructions like identity-based encryption, broadcast encryption and distributed key generation are revealed in Chapter 3.

%%% Local Variables: 
%%% mode: latex
%%% TeX-master: "thesis"
%%% End: 

\chapter{Cryptographic Building Blocks}
\label{cha:2}
This chapter overviews the cryptographic building blocks used to design the encryption mechanism for online social networks proposed in this thesis.

The structure of this chapter is the following. An introduction is given to public key infrastructures and their drawbacks. Then, identity-based encryption (IBE) is overviewed as an alternative to the existing public key infrastructures along with its drawbacks and advantages, the different security definitions and the evolution of IBE in literature. This is followed by an elaborate discussion on broadcast encryption (BE) and secret sharing. Finally, distributed key generation is described as a possible solution to the inherent key escrow problem of IBE.

\section{Public Key Infrastructures}
Asymmetric cryptography assigns users a key pair $\left< sk_i, pk_i \right>$ to allow secure communication between parties who never met. However, a trusted party is required which verifies the binding of an identity to a public key to prevent impersonation. The infrastructure authenticating all public key values is called the Public Key Infrastructure (PKI) as defined by Definition~\ref{def:pki}. However, PKI systems only shift the problem from trusting the users to trusting their keys. For example, if Eve could make the PKI system believe that her own public key $pk_{Eve}$ actually represents the public key of Alice $pk_{Alice}$, Eve would be able to read all Alice's confidential communication as she obviously has the private key $sk_{Eve}$ corresponding to $pk_{Eve}$. Therefore, it is important that public key systems rely on an architecture that authenticates whether key pairs belong to the claimed owner. In practice this is mostly achieved with the help of certification authorities or a web of trust.

\subsection{Certification Authorities}
In a traditional PKI system, all entities in the system trust a central party called the \textit{Certification Authority} (CA). It is the CA that guarantees public keys belong to the claimed owner. CA infrastructures are standardised in X.509~\cite{rfc4158}.

Suppose Alice wants to start using a key pair $\left< pk_A, sk_A \right>$, she has to authenticate herself with the CA by following correctly a protocol that confirms Alice's identity, usually over offline channels. Once Alice is authenticated with the CA, Alice sends the public key $pk_A$ to the CA along with a proof showing that Alice also owns the corresponding private key $sk_A$. This ''proof of correct possession'' often takes the form of a signature $S_{sk_A} \left( pk_A \right)$ generated by the private key $sk_A$ on the public key $pk_A$.

Once the CA is convinced of the authenticity of Alice's public key, it distributes a certificate approving that $pk_A$ effectively belongs to Alice. To avoid forged certificates, the CA signs Alice's certificate with its private key $sk_{CA}$. Anyone doubting the authenticity of the public key $pk_A$ can get convinced $pk_A$ effectively belongs to Alice by checking the signature of the CA with the CA's public key $pk_{CA}$.

In practice, CAs often approve the trustworthiness of other CAs by issuing certificates on their signing keys. In this way, often highly complex hierarchical architectures are achieved that boil down to the trust in one signing key of the highest authority. This puts heavy requirements on the CA's infrastructure as a compromised CA signing key can break the system completely. Indeed, a compromised signing key would allow to sign certificates of unauthenticated public keys or even certificates of public keys that belong to malicious entities.

If an entity's private key is lost or leaked to a third party, it can be revoked by the CA. CAs achieve this by periodic publication of \textit{revocation lists}. These revocation lists contain all compromised public keys. Consequently, users relying on a PKI should always verify these continuously growing lists before trusting a keypair. Thereby, revocation lists not only make the system less transparent, they also impose high demands on the infrastructure of entities relying on the PKI.

Suppose Alice's private key gets compromised due to Eve stealing her hard drive. Alice authenticates with the CA and sends a complaint stating her private key is no longer confidential. In return, the CA puts Alice's public key on the revocation list. The next time Bob wants to send a message to Alice, he first verifies the revocation list to conclude the last version of Alice's public key is no longer in use. However, the revocation list of the CA continuously grows with every additional hard drive Eve can steal thereby increasing the required time to download and verify the revocation list.

To partially get around the issue of revocation lists, certificates contain an expiration date. After expiration, a certificate should no longer be trusted. However, this requires keypair owners to contact CAs more frequently to sign new certificates each time the previous one has expired. Clearly, this puts a high computational demand on the authentication procedure of the CAs as well.

\subsection{OpenPGP and Web of Trust}
\label{sec:web_of_trust}
An alternative to the traditional PKI setting relying on CAs is a \textit{web of trust}. In a web of trust as originally proposed by Zimmerman, any entity can rate the trustworthiness of a public key. For example, if Bob receives Alice's public key personally during a date, the public key can be considered more trustworthy than when Bob receives Alice's key via e-mail. Web of trust systems allow users to vet for the authenticity other users' keys in the system. A standardised web of trust system is OpenPGP~\cite{rfc4880}.

The major advantage of a web of trust is that there no longer needs to be a CA with highly secure infrastructure as the publication of certificates now becomes a shared responsibility.

However, web of trust infrastructures present some issues, such as usability~\cite{art:WhittenT99}. In addition, users are now required to judge for themselves whether they can trust a public key or not. This gives more responsibility to users than most of them can handle without proper knowledge of the consequences to their actions.

\section{Identity-Based Encryption}
The concept of identity-based cryptography was proposed by Shamir~\cite{art:Shamir84} in 1984. In identity-based cryptography any string can be a valid public key for encryption or signature schemes thereby eliminating the need for digital certificates. Identity-based cryptography proves to be particularly elegant if the public key is related to an attribute that uniquely identifies the identity of the user like an e-mail address, an IP address or a telephone number. Consequently, identity-based cryptography reduces system complexity and the cost for establishing and managing the Public Key Infrastructure~(PKI)~\cite{art:BaekNSS04}. 


\subsection{Definition}
A generic Identity-Based Encryption (IBE) scheme is composed of four probabilistic polynomial time algorithms~\cite{art:BonehF01}:
\begin{description}
    \item[\texttt{IBE.Setup($1^{\lambda}$)}] On input of a security parameter $\lambda$, outputs a master secret $sk_{msk}$ and public parameters $params$.
    \item[\texttt{IBE.Extract($params, sk_{msk}, \id{}$):}] Takes public parameters $params$, the master secret $sk_{msk}$, and an \id{} as input and returns the private key $s_{\id{}}$ corresponding to the identity \id{}.
    \item[\texttt{IBE.Encrypt($params, \id{}, m$):}] Returns the encryption $c$ of the message $m$ on the input of the public parameters $params$, the \id{}, and the arbitrary length message $m$.
    \item[\texttt{IBE.Decrypt($s_{\id{}}, c$):}] Decrypts the ciphertext $c =$ \texttt{IBE.Encrypt}($params, \id{}, m$) back to the message $m$ on input of the private key $s_{\id{}}$ corresponding to the receiving identity \id{}.
\end{description}

\begin{figure}[ht]
    \begin{center}
    \scalebox{0.78}{
        \begin{tikzpicture}[auto, node distance=1mm, align=center,
            block/.style={rectangle,text width=6em,text centered,minimum height=11mm},
            line/.style={draw,very thick, ->},
            line2/.style={draw,very thick, <->},
            leg/.style={text centered},
            ]
            %\draw[help lines] (-6,-5) grid (8,3);
            \path
                % Images
                (-0.5,3) node [block] (pkg) {\includegraphics[scale=0.2]{img/pkg.png}}
                (-4,0) node [block] (alice) {\includegraphics[scale=0.2]{img/alice.png}}
                (4,0) node [block] (bob) {\includegraphics[scale=0.2]{img/bob.png}}
                % Text
                (-0.5, 0.5) node [leg, color=cyan] (c) {$c$}
                (1.5, 1.2) node [leg] (id_bob) {$s_{\id{Bob}}$}
                (4.3,2) node [leg] (bob_authenticates) {4. Bob authenticates as \id{Bob}}
                ;
                
       \node[below=of pkg] {\textbf{PKG}};
       \node[below=of alice] {\textbf{Alice} \\ 3. $c \leftarrow $\texttt{IBE.Encrypt($params, \id{Bob}, m$)}};
       \node[below=of bob] {\textbf{Bob} \\ 6. $m \leftarrow $\texttt{IBE.Decrypt($s_{\id{Bob}}, \id{Bob}, c$)}};
       \node[above=of pkg,align=left] {
               1. $\left< sk_{msk}, params \right> \leftarrow $\texttt{IBE.Setup($1^{\lambda}$)} \\
               2. publish $params$ \\
               \\
               5. $s_{\id{Bob}} \leftarrow $\texttt{IBE.Extract($params, sk_{msk}, \id{Bob}$)}
               };
       \begin{scope}[every path/.style=line]
        \path[color=cyan] (alice.east) -- (bob.west);
        \path (2.8,1) -- (1,2.2);
        \path (0.9,2.1) -- (2.7,0.9);
       \end{scope}


        \end{tikzpicture}
    }
    \end{center}
    \caption{Generic identity-based encryption scheme. The blue arrow denotes an insecure channel that can be eavesdropped.}
    \label{fig:generic_ibe_scheme}
\end{figure}

Figure~\ref{fig:generic_ibe_scheme} illustrates the IBE generic algorithms. A trusted Public Key Generator (PKG) generates a master private key $sk_{msk}$ and public parameters $params$ on input of the security parameter $\lambda$. Next, the PKG publishes the public parameters $params$ while storing $sk_{msk}$ preferably in encrypted format on a local disk. If Alice wants to send a message $m$ to Bob, it suffices for her to know the public parameters $params$ and the id $\id{Bob}$, uniquely identifying Bob. Then, Alice encrypts the message to a ciphertext $c$ that is sent over an insecure channel to Bob. On receipt of the ciphertext, Bob authenticates to the PKG over a secure channel to request his private key $sk_{\id{Bob}}$. Subsequently, the PKG generates the private key $sk_{\id{Bob}}$ corresponding to Bob's identity $\id{Bob}$ on input of the master secret key $sk_{msk}$, Bob's id $\id{Bob}$ and public parameters $params$. Subsequently, the PKG sends $sk_{\id{Bob}}$ back again over a secure channel. Bob has now all the required information to decrypt the ciphertext $c$ to its original plaintext message $m$.

\subsection{Comparison with PKI Schemes}
\label{sec:pros_and_cons_of_ibe}
Now we turn to overview the main advantages and disadvantages of generic IBE schemes when compared to more traditional PKI systems.

\subsubsection{Disadvantages}
\paragraph{Single point of failure:}
The PKG generates every private key $sk_{\id{}}$ in the system thereby creating a single point of failure. If a PKG disconnects due to an excessive amount of extraction requests, new users can no longer receive their private keys. However, users already owning a secret key can continue decrypting ciphertexts since this requires no additional communication with the PKG. A PKI system often consists of a hierarchy of CAs. Consequently, an offline CA only takes down a specific part of the PKI. However, an offline CA can no longer issue certificates either.

\paragraph{Key escrow:}
The PKG is required to be trusted since it learns $sk_{\id{}}$ for every entity in the system. A malicious PKG server could use this information to start eavesdropping on the insecure channel between Alice and Bob (the blue arrow in Figure~\ref{fig:generic_ibe_scheme}) while decrypting all ciphertexts that are being sent over. The undesired property that private keys have to be shared with a trusted third party is often called \textit{key escrow} in literature~\cite{art:AbelsonHARBMBJBMDWGJNGRLSISB97}. Since traditional PKIs only authenticate key pairs, key escrow is not an issue.

\paragraph{Revocation of public keys:}
Generic IBE schemes do not support revocation of public keys. However, Bob's private key $sk_{\id{Bob}}$ can still get compromised if he is careless with its storage. In fact, the research community has been focused on the revocation of IBE keys extensively~\cite{art:BoldyrevaGK12,art:BonehDTW01,art:HanaokaHSI05,art:LibertQ03}. Key revocation often requires additional infrastructure that complicates the elegancy of the currently proposed IBE scheme. The major drawback of revoking Bobs key is that Bob can no longer receive encrypted messages because his public key is part of his identity. Therefore, a pragmatic solution to this issue could be to append expiration dates to the public keys. Consequently, public keys will only be valid for a limited amount of time thereby restricting the damage that could be done with a compromised private key~\cite{art:BonehF01}. On the other hand, traditional PKIs publish revocation lists. Although these revocation list are a burden to the complexity of the PKI's infrastructure, they support a necessary feature for practical key management systems. 

\subsubsection{Advantages}
\paragraph{Complexity of the system:}
Only one PKG suffices to realise the IBE scheme, which relaxes expensive infrastructure requirements on the system. Due to the support of revocation lists and the often hierarchical organisation, PKI systems are complex structures with a high amount of redundancy.

\paragraph{User friendliness:}
Users who have no background on cryptographic primitives no longer have to make conscious decisions on key lengths or the randomness of their keys. In an IBE system a public key is a string available to anyone in the system. This is generally conceived as more transparent to users lacking a background on cryptography. An average user knows what a username or an e-mail address represents and to whom it belongs while an authenticated public key is generally not a familiar concept.

\paragraph{Opt-in by default:}
Another useful property of an IBE scheme is that a recipient is not required to actively subscribe to a hierarchy of CAs neither a web of trust before a sender can start sending him messages. In this way, the possibility to send encrypted messages becomes inherently part of any system in which the users are assigned unique identifiers. This is particularly useful in systems where the majority of the users has no knowledge about cryptographic primitives. Users do no longer need to generate a key pair neither subscribe to a third party infrastructure. It suffices to recall how their connections can be uniquely identified in the system to learn their public keys.

\subsection{Security of IBE}
IBE schemes follow similar security notions as generic public key systems. Therefore, definitions of security are often subtle as different levels of security can be distinguished. In literature, the security notions mostly considered are \textit{indistinguishability under chosen plaintext attack} (IND-CPA) and \textit{indistinguishability under chosen ciphertext attack} (IND-CCA). Anonymity of the encryption scheme is an additional property of the scheme that is often desired~\cite{art:BellareBDP01}.

For a more in depth discussion on IND-CPA and IND-CCA, the reader is referred to Boneh and Franklin~\cite{art:BonehF01}, whereas for a more formal description of ciphertext anonymity the reader is referred to Abdalla et al.~\cite{art:AbdallaBCKKLMNPS05}.

\subsubsection{Indistinguishability Under Chosen Plaintext Attack}
Indistinguishability under chosen plaintext attack (IND-CPA) is described by the negligible advantage an adversary has in trying to distinguish which of both given plaintext messages $m_0$ and $m_1$ generated a ciphertext $c$. It captures the notion of \textit{semantic security}, i.e. that any ciphertext $c$ should not give more information about the original plaintext $m$ than any other random binary string of the same length.

IND-CPA is best defined with the help of a game that challenges the adversary. The advantage of the adversary in winning the IND-CPA game illustrated in Game~\ref{game:ind_cpa_game}, is defined as
\begin{equation*}
 Adv = \lvert \textrm{Pr} \left[ b = b' \right] - \frac{1}{2} \rvert
\end{equation*}

If the adversary has negligible advantage trying to win the IND-CPA game, the IBE system is said to be IND-CPA secure. More formally, an IBE system is IBE-IND-CPA secure if for every adversary with advantage $Adv$ in winning the IBE-IND-CPA game illustrated in Game~\ref{game:ind_cpa_game} there exists a negligible function $\mu \left( \lambda \right)$ such that $Adv \leq \mu \left( \lambda \right)$.

\begin{game}
\caption{Generic IBE-IND-CPA Game~\cite{thesis:Alfredo08}}
\label{game:ind_cpa_game}
\begin{description}
 \item \textbf{Goal}: An adversary is challenged by a game to check the IND-CPA security of an IBE scheme.
 
 \item \textbf{Result}: This IBE-IND-CPA Game helps to define the concept of IND-CPA security for IBE schemes.
\end{description}

 \begin{enumerate}
  \item The challenger runs $\left< sk_{msk}, params\right> \leftarrow$ \texttt{IBE.Setup($1^{\lambda}$)} and returns $params$ to the adversary.
  \item \label{item:firs_oracle_query} The adversary can start querying an oracle $O_{Extract} \left( \id{i} \right)$ that returns a private key $sk_{\id{i}} \leftarrow$ \texttt{IBE.Extract($params, sk_{msk}, \id{}$)} corresponding to an adversary defined identity $\id{i}$.
  \item The adversary picks two equal length plaintext messages $m_0$ and $m_1$ and an identity $\id{encrypt}$. The adversary honestly passes $\left< m_0, m_1, \id{encrypt} \right>$ to the challenger.
  \item The challenger picks a random bit $b$ and executes \\ $c \leftarrow$ \texttt{IBE.Encrypt($params, \id{encrypt}, m_b$)}. The challenger gives $c$ to the adversary.
  \item \label{item:second_oracle_query} The adversary continues querying the oracle $O_{Extract} \left( \id{i} \right)$ adaptively.
  \item The adversary outputs a bit $b'$ based on the ciphertext $c$. If $b = b'$ the adversary wins the game. If $b \neq b'$ or if the adversary queried the oracle $O_{Extract} \left( \id{i} \right)$ with $\id{i} = \id{encrypt}$ during step~\ref{item:firs_oracle_query} or step~\ref{item:second_oracle_query}, the adversary loses the game.
 \end{enumerate}
\end{game}

\subsubsection{Indistinguishability Under Chosen Ciphertext Attack}
Indistinguishability under chosen ciphertext (IND-CCA) is a more demanding level of security. Therefore, an algorithm that is IND-CCA secure is considered more secure than an IND-CPA secure algorithm. IND-CCA security implies that an adversary has no advantage in trying to distinguish which of both given plaintext messages $m_0$ and $m_1$ generated a ciphertext $c$ even if the adversary has access to a list of (plaintext, ciphertext)-tuples.

IND-CCA is defined with the help of a game that challenges an adversary similar to the IND-CPA game. Compared to the IND-CPA game, the IND-CCA game contains two additional steps in which the adversary gets access to another oracle. %If the adversary has negligible advantage trying to win the IND-CCA game from Game~\ref{game:ind_cca_game}, the IBE system is said to be IND-CCA secure. 
The advantage of the adversary in winning the IND-CCA game illustrated in Game~\ref{game:ind_cca_game}, is defined as
\begin{equation*}
 Adv = \lvert \textrm{Pr} \left[ b = b' \right] - \frac{1}{2} \rvert
\end{equation*}

If the adversary has negligible advantage trying to win the IND-CCA game, the IBE system is said to be IND-CCA secure. More formally, an IBE system is IBE-IND-CCA secure if for every adversary with advantage $Adv$ in winning the IBE-IND-CCA game illustrated in Game~\ref{game:ind_cca_game} there exists a negligible function $\mu \left( \lambda \right)$ such that $Adv \leq \mu \left( \lambda \right)$.

In literature a distinction is often made between a \textit{non-adaptive} case (IND-CCA1) and an \textit{adaptive} case (IND-CCA2) of IND-CCA. In the non-adaptive case, step 6 from Game~\ref{game:ind_cca_game} is not allowed. More precisely, an IBE scheme that satisfies Game~\ref{game:ind_cca_game} is said to be IND-CCA2 secure.

\begin{game}
\caption{Generic IBE-IND-CCA Game~\cite{thesis:Alfredo08}}
\label{game:ind_cca_game}
\begin{description}
 \item \textbf{Goal}: An adversary is challenged by a game to check the IND-CCA security of an IBE scheme.
 
 \item \textbf{Result}: This IBE-IND-CCA Game helps to define the concept of IND-CPA security for IBE schemes.
\end{description}

 \begin{enumerate}
  \item The challenger runs $\left< sk_{msk}, params\right> \leftarrow$ \texttt{IBE.Setup($1^{\lambda}$)} and returns $params$ to the adversary.
  \item \label{item:cca_first_oracle_query} The adversary can start querying an oracle $O_{Extract} \left( \id{i} \right)$ that returns a private key $sk_{\id{i}} \leftarrow$ \texttt{IBE.Extract($params, sk_{msk}, \id{}$)} corresponding to an adversary defined identity $\id{i}$.
  \item \label{item:cca_first_decryption_query} The adversary can start querying another oracle $O_{Decrypt} \left( sk_{\id{i}}, c_j \right)$ that returns a plaintext $m_j \leftarrow$ \texttt{IBE.Decrypt($sk_{\id{i}}, c_j$)} corresponding to an adversary defined ciphertext $c_j$ and identity $\id{i}$.
  \item The adversary picks two equal length plaintext messages $m_0$ and $m_1$ and an identity $\id{encrypt}$. The adversary honestly passes $\left< m_0, m_1, \id{encrypt} \right>$ to the challenger.
  \item The challenger picks a random bit $b$ and executes \\ $c \leftarrow$ \texttt{IBE.Encrypt($params, \id{}, m_b$)}. The challenger gives $c$ to the adversary.
  \item \label{item:cca_second_oracle_query} The adversary continues querying the oracle $O_{Extract} \left( \id{i} \right)$ adaptively.
  \item \label{item:cca_second_decryption_query} The adversary continues querying the oracle $O_{Decrypt} \left( sk_{\id{i}}, c_j \right)$ adaptively.
  \item The adversary outputs a bit $b'$ based on the ciphertext $c$. If $b = b'$ the adversary wins the game. Otherwise, the adversary loses the game. If the adversary queried the oracle $O_{Extract} \left( \id{i} \right)$ with $\id{i} = \id{encrypt}$ during step~\ref{item:cca_first_oracle_query} or step~\ref{item:cca_second_oracle_query} or if the adversary queried the oracle $O_{Decrypt} \left( sk_{\id{i}}, c_j \right)$ with $c_j = c$ during step~\ref{item:cca_first_decryption_query} or step~\ref{item:cca_second_decryption_query}, the adversary loses the game as well.
 \end{enumerate}
\end{game}

\subsubsection{Anonymous Identity-Based Encryption}
An IBE scheme is called anonymous (ANO-IBE) when the ciphertext does not leak the identity of the recipient. In the overview illustrated in Figure~\ref{fig:generic_ibe_scheme}, this implies that no eavesdropper on the insecure channel between Alice and Bob could derive that Bob is the recipient based on the information in the ciphertext $c$ alone~\cite{art:BoyenW06}.

ANO-IBE is defined with the help of a game that challenges an adversary similar to the IND-CPA game. Similar to IND-CCA and IND-CPA security, an IBE system is said to be anonymous if the adversary has negligible advantage trying to win the ANO-IBE game in Game~\ref{game:ano_ibe}. Again, the advantage of the adversary in winning the IND-CCA game illustrated in Game~\ref{game:ind_cca_game}, is defined as
\begin{equation*}
 Adv = \lvert \textrm{Pr} \left[ b = b' \right] - \frac{1}{2} \rvert
\end{equation*}

More formally, an IBE system is ANO-IBE secure if for every adversary with advantage $Adv$ in winning the ANO-IBE game illustrated in Game~\ref{game:ano_ibe} there exists a negligible function $\mu \left( \lambda \right)$ such that $Adv \leq \mu \left( \lambda \right)$.

Gentry~\cite{art:Gentry06} presents the first scheme which combines the notions of IND-CPA and IND-CCA with ANO-IBE. Therefore, a system is then said to be IND-ANO-CPA secure or IND-ANO-CCA secure if it satisfies a modified version of the game in Game~\ref{game:ano_ibe}. For a more detailed discussion on the topic the reader is referred to the original paper~\cite{art:Gentry06}.

\begin{game}
\caption{Generic ANO-IBE Game~\cite{thesis:Alfredo08}}
\label{game:ano_ibe}
\begin{description}
 \item \textbf{Goal}: An adversary is challenged by a game to check the ANO-IBE security of an IBE scheme.
 
 \item \textbf{Result}: This ANO-IBE Game helps to define the concept of ANO-IBE security for IBE schemes.
\end{description}
 \begin{enumerate}
  \item The challenger runs $\left< sk_{msk}, params\right> \leftarrow$ \texttt{IBE.Setup($1^{\lambda}$)} and returns $params$ to the adversary.
  \item \label{item:first_ano_query} The adversary can start querying an oracle $O_{Extract} \left( \id{i} \right)$ that returns a private key $sk_{\id{i}} \leftarrow$ \texttt{IBE.Extract($params, sk_{msk}, \id{i}$)} corresponding to an adversary defined identity $\id{i}$.
  \item The adversary picks a plaintext message $m$ and an identity \id{encrypt}. The adversary honestly passes $\left< m, \id{encrypt} \right>$ to the challenger.
  \item The challenger picks a random bit $b$ and computes \\ $c \leftarrow$ \texttt{IBE.Encrypt($params, \id{encrypt}, m$)} if $b=0$. If $b=1$, the challenger computes $c \leftarrow$ \texttt{IBE.Encrypt($params, \id{encrypt}, r$)} where $r$ is a random bit sequence with the same length as the message $m$. The challenger gives $c$ to the adversary.
  \item \label{item:second_ano_query} The adversary continues querying the oracle $O_{Extract} \left( \id{i} \right)$ adaptively.
  \item The adversary outputs a bit $b'$ based on the ciphertext $c$. If $b = b'$ the adversary wins the game. If $b \neq b'$ or if the adversary queried the oracle $O_{Extract} \left( \id{i} \right)$ with $\id{i} = \id{encrypt}$ during step~\ref{item:first_ano_query} or step~\ref{item:second_ano_query}, the adversary loses the game.
 \end{enumerate}
\end{game}

% Uiteenzetting hoe we tot BF-IBE DKG komen:
%    -> BF-IBE is anoniem, BB is NIET anoniem
%            => zie winter lecture op 
%                  https://www.youtube.com/watch?v=Tt7cJnZDth0&index=8&list=PLXF_IJaFk-9C4p3b2tK7H9a9axOm3EtjA
%    -> SK-IBE en BB vereisen communicatie tussen DKGs
%       tijdens extractie stap.
%    -> SK-IBE berust op BDHI veronderstelling = niet leuk
%    -> BB is vorm van HIBE => totaal aantal gebruikers moet op voorhand worden vastgelegd = niet leuk
%    -> BB is enkel selective-ID CCA secure = niet leuk
\subsection{Overview}
\label{sec:evolution_of_ibe}
Although Shamir~\cite{art:Shamir84} easily constructed an identity-based signature scheme based on RSA in 1984, the practical use of IBE remained an open problem until the introduction of bilinear maps. Boneh and Franklin~\cite{art:BonehF01} proposed the first practically usable IBE scheme based on the Weil pairing, however, the security proof still relies on the random oracle assumption. At the same time, Sakai and Kasahara~\cite{art:SakaiOK01} proposed a different IBE scheme independently from Boneh and Franklin. The scheme from Sakai and Kasahara initially received less attention, because the original presentation is in Japanese and lacking a security proof. Subsequently, Sakai and Kasahara~\cite{art:SakaiK03} proposed an extended version of their original scheme which is proven to be IND-CCA secure in the random oracle model by Chen et al.~\cite{art:ChenC05} 

Canetti et al.~\cite{art:CanettiHK03} introduced the first secure IBE scheme without relying on the random oracle model. Nevertheless, the attacker model in~\cite{art:CanettiHK03} requires the adversary to declare upfront which identity \id{} is targeted during step 5 of the CCA Game (Algorithm~\ref{game:ind_cca_game}) and step 4 of the CPA Game. Therefore, the scheme by Boneh and Franklin~\cite{art:BonehF01} is considered more secure as attackers can adaptively choose the targeted identity. Later, Boneh and Boyen~\cite{art:BonehB04} presented a variant to~\cite{art:CanettiHK03} which also realises only selective ID security.

Waters~\cite{art:Waters05} is the first to present a scheme that is IND-CCA secure in the standard model. Drawback of the scheme from Waters~\cite{art:Waters05} is that it requires large public parameters. Gentry~\cite{art:Gentry06} proposes a more efficient alternative to this scheme in the standard model while achieving shorter public parameters. However, the scheme from Gentry relies on a complicated hardness assumption called q-BDHE. It is only after the introduction of the Dual System paradigm by Waters~\cite{art:Waters09} in 2009 that IND-CCA security can be achieved in the standard model based on reasonable assumptions. De Caro et al.~\cite{art:CaroIP10} are the first to define an IND-ANO-CCA secure IBE scheme on the Dual System construction of Waters~\cite{art:Waters09}. % Note that paper from Lewko and Waters only considers HIBE and no IBE scheme.

Although all these references contributed to the evolution of IBE, not all of these schemes are ANO-IBE. The IBE scheme from Boneh and Franklin~\cite{art:BonehF01} is IND-ANO-CCA secure since IBE systems in the random oracle model are ANO-IBE. In the standard model, it appeared to be harder to construct ANO-IBE schemes at first sight, e.g. it can be proven that the scheme from Boneh and Boyen~\cite{art:BonehB04} is not anonymous in its original form. The scheme from Gentry~\cite{art:Gentry06} was the first anonymous IBE scheme in the standard model. Boyen and Waters~\cite{art:BoyenW06} published almost synchronously another IBE scheme in the standard model that is also IND-ANO-CCA secure. In 2010, Ducas~\cite{art:Ducas10} showed that even schemes that were first considered not anonymous like the one from Boneh and Boyen~\cite{art:BonehB04} but also~\cite{art:BonehBG05,art:Waters05} can be proven anonymous when relying on asymmetric pairings thereby making anonymity a more common property in IBE schemes.

\subsection{Most Attractive IBE Schemes}
In the standard model mainly the anonymous IBE constructions from Gentry~\cite{art:Gentry06} and De Caro et al.~\cite{art:CaroIP10} have the most satisfying properties. However, IBE constructions in the standard model often come at the cost of higher computational requirements~\cite{art:Boyen08}. Certainly the scheme from De Caro demands a higher amount of computational resources since it relies on composite order groups. Although methods~\cite{art:Freeman10,art:Lewko12} have been developed to convert IBE schemes from composite order groups to single order prime groups, these methods do not apply to the scheme from De Caro et al.~\cite{art:LeeL10}

From all schemes discussed in Section~\ref{sec:evolution_of_ibe} the ones initially developed by Boneh and Franklin~\cite{art:BonehF01} and Sakai and Kasahara~\cite{art:SakaiK03} are the most attractive ones in the random oracle model because of their anonymity and non-selective security. Consequently, it is not a coincidence that both schemes have found description in an informational RFC document. Sakai and Kasahara IBE is described in RFC 6508~\cite{rfc6508} and RFC 6509~\cite{rfc6509}. Boneh and Franklin IBE can be found in RFC 5409~\cite{rfc5409}. 

The ANO-IND-CCA secure scheme from Boneh an Franklin~\cite{art:BonehF01} is included in Algorithm~\ref{alg:full_indent} since it is valuable for the remainder of the text.

%\begin{algorithm}
%\caption{IND-ANO-CPA Boneh and Franklin IBE~\cite{art:BonehF01}}
%\label{alg:basic_indent}
%\begin{description}
% \item \textbf{Goal}: Alice wants to send an IBE encrypted message to Bob.
 
% \item \textbf{Result}: Alice sends an IBE encrypted ciphertext $c$ that is successfully decrypted by Bob.
%\end{description}
% \begin{enumerate}
%  \item \texttt{Setup($1^{\lambda}$)}: Let $\lambda$ be the security parameter for a security level of $l$ bits.
%  \begin{enumerate}
%   \item Execute setup algorithm $\left< q, G_1, G_2, e: G_1 \times G_2 \rightarrow G_T, P \in G_1 \right> \leftarrow \mathcal{G} \left( 1^{\lambda} \right)$ to generate the parameters
%    \begin{enumerate}
%     \item A large prime $q$
%     \item Gap groups $G_1$ and $G_2$ of order $q$
%     \item An admissible bilinear map $e: G_1 \times G_2 \rightarrow G_T$
%     \item A random generator $P \in G_1$
%   \end{enumerate}
%   \item Choose a uniformly random $sk_{msk} \in \mathbb{Z}^{*}_q$ and calculate
%   \begin{equation*}
%    P_{pub} = sk_{msk} P
%   \end{equation*}
%   \item Choose cryptographic hash functions
%    \begin{enumerate}
%     \item $H_1: \{ 0,1  \}^* \rightarrow G_1$
%     \item $H_2: G_2 \rightarrow \{ 0,1 \}^l$
%    \end{enumerate}
%  \end{enumerate}
%  \item \texttt{Extract($params, sk_{msk}, \id{}$)}:
%   \begin{enumerate}
%    \item Compute $Q_{\id{}} = H_1 \left( \id{} \right) \in G_1$
%    \item Set the private key of $\id{}$ to $sk_{\id{}} = sk_{msk} Q_{\id{}}$
%   \end{enumerate}
%   \item \texttt{Encrypt($params, \id{}, m$)}:
%   \begin{enumerate}
%    \item Compute $Q_{\id{}} = H_1 \left( \id{} \right)$
%    \item Choose a random $r \in Z_q$
%    \item Encrypt the plaintext message $m$ to the ciphertext $c$ as
%    \begin{equation*}
%     c = \left< rP, m \oplus H_2 \left( g_{\id{}^r} \right) \right> = \left< U, v \right> \; \; \textrm{with} \; \; g_{\id{}} = e \left( Q_{\id{}}, P_{pub} \right) \in G_T
%    \end{equation*}
%   \end{enumerate}
%   \item \texttt{Decrypt($sk_{\id{}}, c$)}: Decrypt the ciphertext $c$ back to the plaintext message $m$ as
%   \begin{equation*}
%    m = v \oplus H_2 \left( e \left( sk_{\id{}}, U \right) \right)
%   \end{equation*}
% \end{enumerate}
%\end{algorithm}
%\makebox[10cm][s]{

\afterpage{ % execute argument of this command *after* end of current page
\clearpage
\begin{algorithm}
\caption{IND-ANO-CCA Boneh and Franklin IBE~\cite{art:BonehF01}}
\label{alg:full_indent}
 \textbf{Goal}: Alice wants to send an IBE encrypted message to Bob. \\
 \textbf{Result}: Alice sends an IBE encrypted ciphertext $c$ that is successfully decrypted by Bob.
 \begin{enumerate}
  \item \texttt{Setup($1^{\lambda}$)}: 
  \begin{enumerate}
   \item Execute setup algorithm $\left< q, G_1, G_2, e: G_1 \times G_2 \rightarrow G_T, P \in G_1 \right> \leftarrow \mathcal{G} \left( 1^{\lambda} \right)$ to generate the parameters
    \begin{enumerate}
     \item A large prime $q$
     \item Gap groups $G_1$ and $G_2$ of order $q$
     \item An admissible bilinear map $e: G_1 \times G_2 \rightarrow G_T$
     \item A random generator $P \in G_1$
   \end{enumerate}
   \item Choose a uniformly random $sk_{msk} \in \mathbb{Z}^{*}_q$ and compute $P_{pub} = sk_{msk} P$

   \item Choose cryptographic hash functions
    \begin{enumerate}
     \item $H_1: \{ 0,1  \}^* \rightarrow G_1$
     \item $H_2: G_2 \rightarrow \{ 0,1 \}^l$
     \item $H_3: \{ 0,1 \}^l \rightarrow \left( 0,1 \right)^l$
    \end{enumerate}
  \end{enumerate}
  \item \texttt{Extract($params, sk_{msk}, \id{}$)}:
   \begin{enumerate}
    \item Compute $Q_{\id{}} = H_1 \left( \id{} \right) \in G_1$
    \item Set the private key of $\id{}$ to $sk_{\id{}} = sk_{msk} Q_{\id{}}$
   \end{enumerate}   \item \texttt{Encrypt($params, \id{}, m$)}:
   \begin{enumerate}
    \item Compute $Q_{\id{}} = H_1 \{ \id{} \}$
    \item Choose a random $sigma \in \left( 0,1 \right)^l$
    \item Compute $r = H_3 \{ sigma, m \}$
    \item Encrypt the plaintext message $m$ to the ciphertext $c$ as
    \begin{equation*}
     \begin{split}
      c = \left< rP, sigma \oplus H_2 \left( g_{\id{}}^r\right), m \oplus H_3 \left( sigma \right) \right> = \left< U, v, w \right> \\ 
     \; \; \textrm{with} \; \; g_{\id{}} = e \left( Q_{\id{}}, P_pub \right) \in G_T
     \end{split}
    \end{equation*}
   \end{enumerate}
   \item \texttt{Decrypt($sk_{\id{}}, c$)}: Decrypt the ciphertext $c$ back to the plaintext message $m$ as follows
   \begin{enumerate}
    \item Compute $sigma = v \oplus H_2 \left( e \left( sk_{\id{}}, U \right) \right)$
    \item Compute $m = w \oplus H_3 \left( sigma \right)$
    \item Set $r = H_3 \left( sigma, m \right)$. Test that $U = rP$. If not, reject the ciphertext.
    \item Output $m$ as the decryption of $c$
    \end{enumerate}
 \end{enumerate}
\end{algorithm}
\clearpage
}

\section{Broadcast Encryption}
Another relevant aspect of encryption in OSNs is how one encrypted message can be securely broadcasted to multiple users. To this means, broadcast encryption (BE) was introduced by Fiat and Naor~\cite{art:FiatN93}, as a public-key generalisation to a multi user setting. In particular, a BE scheme allows a user to encrypt a message $m$ to a subset $\mathcal{S}$ of users in a public key system, such that, only users in the set $\mathcal{S}$ are able to decrypt the message. The computational overhead of a BE is generally bound to the ciphertext and the number of recipients.

\subsection{Definition}
A generic Broadcast Encryption (BE) scheme is composed of four probabilistic polynomial time algorithms:

\begin{description}
    \item[\texttt{BE.Setup($1^{\lambda}$)}]: On input of a security parameter $\lambda$, generates the public parameters $params$ of the system.
    \item[\texttt{BE.KeyGen($params$)}]: Returns the public and private key ($pk_i,sk_i$) for each user $i$ while taking the public parameters $params$ into account.
    \item[\texttt{BE.Encrypt($m, \mathcal{S}$)}]: Takes a set of public key values $\mathcal{S}=\{pk_i \ldots pk_{|\mathcal{S}|}\}$ corresponding to users $i$ in the system along with a plaintext message $m$ to generate a corresponding ciphertext $c$.
    \item[\texttt{BE.Decrypt($c, sk_i$):}] Reconstructs $m$ from $c$ using the private key $sk_i$ if the corresponding public key $pk_i \in \mathcal{S}$. Otherwise, return $\bot$.
\end{description}

Note that this definition is stated generically enough to allow all kinds of public keys to be used. Therefore, not only traditional PKIs can benefit from BE schemes, but also IBE schemes in which a public identifier \id{i} serves as a public key $pk_i$.

\subsection{Overview}
\label{sec:evolution_of_be}
The issue of encrypting one message to reach multiple recipients has been widely studied in literature since its first introduction by Fiat and Naor~\cite{art:FiatN93}. This section highlights the most important evolutions of BE in literature, however it only considers the most relevant publications to our goal: achieving user-friendly broadcast encryption for OSNs.

\subsubsection{Broadcast Encryption}
The implementation from Fiat and Naor~\cite{art:FiatN93} requires a ciphertext of size  $O \left( t \log^2 t \log n \right)$ to be~secure against $t$ colluding users. The first fully collusion resistant scheme was proposed by Naor et al.~\cite{art:NaorNL01} thereby making the ciphertext size independent of the number of colluding users. A collusion resistant BE scheme refers to a broadcast encryption scheme that is secure even if all users that are not in the recipient set $\mathcal{S}$ would collaborate. Halevy and Shamir~\cite{art:HalevyS02} further reduce the required ciphertext length for collusion resistant schemes followed by many~\cite{art:DodisF02,art:GoodrichST04,art:LewkoSW08} achieving ciphertext sizes only dependent on the number of revoked users $O \left( r \right)$. Boneh, Gentry and Waters~\cite{art:BonehBG05} are the first to consider utilisation of bilinear maps to realise constant size ciphertexts and $O \left( n \right)$ public keys.

\subsubsection{Identity-Based Broadcast Encryption}
Sakai and Furukawa are the first to define a collusion resistant identity based broadcast encryption (IBBE) scheme in~\cite{art:SakaiF07}. Independently, Delerabl\'{e}e~\cite{art:Delerablee07} realises a similar IBBE scheme and claims to be the first as well.  The size of the public key in both schemes is proportional to the maximum size of the intended set of recipients while realising short ciphertexts and private keys. 

Baek et al.~\cite{art:BaekNSS04} defines an IBBE scheme that requires only one pairing computation, proven secure under the random oracle assumption where the attacker ties himself to a selective-ID attack. Later, Gentry and Waters achieve identity based broadcast encryption with sublinear ciphertexts in~\cite{art:GentryW08}. Their scheme is proven secure against the stronger notion of adaptive security where the attacker can adaptively alter its queries depending on earlier received information. Barbosa and Farshim~\cite{art:BarbosaF05} proposed an identity-based key encapsulation scheme for multiple parties which is an extension of \textit{mKEM} as considered by Smart~\cite{art:Smart04} to the identity-based setting. An mKEM is a Key Encapsulation Mechanism which takes multiple public keys as input. An encrypted message under mKEM consists of an encapsulated session key $k$ and a symmetric encryption $E_k \left( m \right)$ of the plaintext message $m$ under $k$. However, the scheme from Smart~\cite{art:Smart04} is only proven secure under the random oracle assumption.

\subsubsection{Anonymous Broadcast Encryption}
\label{sec:anobe}
All earlier mentioned references describing BE require the intended set of recipients to be published to realise higher efficiency. Barth, Boneh and Waters~\cite{art:BarthBW06} are the first to design a BE scheme that takes the anonymity of the recipient into account.
\begin{defn}[Anonymity]
\label{def:anonymity}
 A BE scheme is said to be \textit{anonymous} if it hides who is included in the recipient set $\mathcal{S}$. That is, no entity inside or outside $\mathcal{S}$ can derive the identity of recipients included in $\mathcal{S}$ from the broadcasted ciphertext.
\end{defn}

The proposed anonymous broadcast encryption (ANOBE) scheme  from Barth, Boneh and Waters~\cite{art:BarthBW06} implies a linear dependency of the ciphertext on the number of recipients and can only be proven secure in the random oracle model. In~\cite{art:LibertPQ12} Libert et al., propose an alternative ANOBE scheme that is proven secure in the standard model. Both~\cite{art:BarthBW06} and~\cite{art:LibertPQ12} propose a tag based system that allows efficient decryption at the cost of making the public master key linear dependent on the total number of users. Krzywiekci et al.~\cite{art:KrzywieckiKK06} propose a scheme that is proportional to the number of revoked users, although the security proof is rather informal. In~\cite{art:YuRL10}, Yu et al. design an architecture that even hides the number of users in the recipient set using Attribute Based Encryption (ABE)~\cite{art:SahaiW04}.

However, ABE requires that all users are assigned attributes such that all users who have sufficient attributes in common can decrypt the message. In networks where the total number of users is large it can be a work intensive task to label each user with the correct attributes.

\subsubsection{Outsider-Anonymous Broadcast Encryption}
The notion of outsider anonymous broadcast encryption is introduced by Fazio and Perera~\cite{art:FazioP12}.
\begin{defn}[Outsider Anonymity]
\label{def:outsider_anonymity}
 A BE scheme is called \textit{outsider anonymous} if the identities of the recipients are known to the other identities in the recipient set $\mathcal{S}$ while remaining secret to other parties of the BE scheme.
\end{defn}

The scheme from Fazio and Perera~\cite{art:FazioP12} relies on IBE to encode where a recipient is positioned in a publicly published tree to achieve sublinear ciphertexts. It is remarkable that sublinear ciphertexts are achieved while attaining recipient anonymity to all users that are outside the intended set of receivers. However, the scheme has the drawback of immediately fixing the total number of users that are allowed in the system. Furthermore, an additional architecture is required to maintain the tree of subscribed users. Although IBE is used, the scheme does not allow to represent public keys of users by their public identifiers, because the public key needs to be the position of a user in the tree structure of the external architecture. In this way, most of the desirable properties of IBE cancel out. Although the scheme from Fazio and Perera does not fit the requirements for user-friendly broadcast-encryption in OSNs, it is useful to remember their definition of outsider-anonymity.

\subsection{Most Attractive BE Schemes}
From the aforementioned schemes the one from Libert et al.~\cite{art:LibertQ03} contains the most attractive properties as it is proven secure in the standard model at almost no reduced computational efficiency. The scheme supports anonymity in both identity-based BE as well as traditional asymmetric cryptosystems.

If anonymity is not an issue, different BE schemes have to be considered depending on the goals of the target application. The scheme from Libert et al.~\cite{art:LibertPQ12} certainly does not have the most desirable properties in non-anonymous BE environments since it can not benefit from higher efficiency due to the recipient being publicly known.

\section{Secret Sharing}
In 1979, both Shamir~\cite{art:Shamir79} and Blakley~\cite{art:Blakley79} independently proposed an algorithm achieving perfect threshold secret sharing.

\begin{defn}[Secret Sharing Scheme]
\label{def:secret_sharing_scheme}
 A \textit{Secret Sharing Scheme} is a cryptographic scheme that divides a secret $s$ into $n$ pieces of data $\sigma_1, \ldots, \sigma_n$ called \textit{shares}. Shares are distributed over $n$ different parties called \textit{shareholders} such that only specific subsets of the distributed shares allow reconstruction of the original secret $s$.
\end{defn}

\begin{defn}[Threshold scheme]
\label{def:threshold_scheme}
 A $\left( t, n \right)$ \textit{threshold scheme} $\left( t \leq n \right)$ is a secret sharing scheme by which a trusted party securely distributes $n$ different shares $\sigma_i$ to $n$ different parties $P_i$ for $1 \leq i \leq n$ such that any subset of $t$ or more different shares $\sigma_i$ easily allows to reconstruct the original secret $s$. Knowledge of $t-1$ or less shares is insufficient to reconstruct the original secret $s$.
\end{defn}

\begin{defn}[Perfect threshold scheme]
\label{def:threshold_scheme}
 A $\left( t, n \right)$ threshold scheme is said to be \textit{perfect} if no subset of fewer than $t$ shareholders can derive any partial information in the information theoretic sense about the original secret $s$ even with infinite computational resources.
\end{defn}

Shamir's solution~\cite{art:Shamir79} was based on polynomial interpolation while Blakley's algorithm~\cite{art:Blakley79} relied on finite geometries. Blakley secret sharing uses more bits than necessary as it describes multidimensional planes. In contrast, Shamir secret sharing requires as many bits for each share as the length of the original secret.


\begin{algorithm}
\caption{Shamir's $\left( t, n \right)$ threshold scheme~\cite{book:handbook_of_applied_cryptography} }
\label{alg:shamirs_threshold_sheme}
\begin{description}
 \item \textbf{Goal}: A dealer $D$ distributes shares of a secret $s$ to $n$ parties.
 
 \item \textbf{Result}: If a subset of at least $t$ out of $n$ shareholders collaborates, they can reconstruct the original secret $s$.
\end{description}

 \begin{enumerate}
  \item \textit{Setup} A dealer $D$ begins with a secret integer $s \geq 0$ it wishes to distribute among $n$ parties
   \begin{enumerate}
    \item $D$ chooses a prime $p > \max \left( s, n \right)$ and defines $a_0 = s$
    \item $D$ selects $t-1$ random, independent coefficients $a_1, \ldots, a_{t-1}, 0 \leq a_j \leq p-1$ defining the random polynomial over $\mathbb{Z}_p$, $f \left( x \right) = \sum^{t-1}_{j=0} a_j x^j$
    \item $D$ computes $\sigma_i = f \left( i \right) \bmod p, 1 \leq i \leq n$ and securely transfers the share $\sigma_i$ to shareholder $P_i$, along with a public index $i$.
   \end{enumerate}
   \item \textit{Reconstruction} Any group of $t$ or more shareholders pool their shares. Their shares provide $t$ distinct points $\left( x, y \right) = \left( i, \sigma_i \right)$ allowing computation of the coefficients $a_j, 1 \leq j \leq t-1$ of $f \left( x \right)$ by Lagrange interpolation. The secret is recovered by calculating
 \begin{equation*}
  f \left( 0 \right) = \sum^t_{i=1}y_i b_i = s \; \; \; \textrm{with} \; \; \; b_i = \prod_{1 \leq j \leq t, j \neq i} \frac{j}{j-i}
 \end{equation*}
 \end{enumerate}
\end{algorithm}

The idea behind Shamir secret sharing is elegant in its simplicity. Shamir secret sharing is based on the principal that any polynomial $f \left( x \right)$ of degree $t-1$ is uniquely defined by $t$ points lying on the polynomial. For example, it is possible to draw only one straight line between 2 different coordinates, a quadratic is fully defined by 3 different coordinates and so on. If the trusted party randomly generates a polynomial of degree $t-1$ it suffices to securely distribute one of $n$ different coordinates on the curve to each party $P_i, 0 \leq i \leq n$. A subset of at least $t$ different shareholders has to collaborate in order to reconstruct the original polynomial by interpolation. For security reasons the polynomial $f \left( x \right)$ is calculated in a finite field modulo a large prime number $p$. The complete mechanism of Shamir's threshold scheme is depicted in Algorithm~\ref{alg:shamirs_threshold_sheme}. The mechanism behind reconstruction in Algorithm~\ref{alg:shamirs_threshold_sheme} is explained because the coefficients of an unknown polynomial $f \left( x \right)$ of degree less than $t$, defined by points $\left( x_i, y_i \right), 1 \leq i \leq t$ are given by the Lagrange interpolation formula
\begin{equation*}
 f \left( x \right) = \sum^t_{i=1}y_i b_i \; \; \; \textrm{with} \; \; \; b_i = \prod_{1 \leq j \leq t, j \neq i} \frac{x-x_j}{x_i-x_j}
\end{equation*}
A proof of this formula is omitted but can be found in~\cite{site:proofwiki_lagrange}.

\subsection{Verifiable Secret Sharing}
Verifiable secret sharing~\cite{art:ChorGMA85} tries to ensure the participating parties that received shares are consistent by providing a verification mechanism. This verification mechanism can either detect an unfair dealer during setup or participants submitting incorrect shares during the reconstruction phase. The first verifiable secret sharing schemes were \textit{interactive}, i.e. interaction between shareholders and the trusted party was required to verify their shares. In \textit{non-interactive verifiable secret sharing} as proposed by Pedersen~\cite{art:Pedersen91} only the trusted party is allowed to send messages to the future shareholders. Shareholders can not communicate with each other neither can they send messages back to the trusted party. Non-interactive verifiable secret sharing is preferred over interactive alternatives as their is no chance of shareholders accidentally leaking too much information.

Popular verifiable secret sharing schemes are Feldman's scheme~\cite{art:Feldman87} and Benaloh's scheme~\cite{art:Benaloh86a}. No further details are given as a basic notion of verifiable secret sharing suffices for the remainder of this text.

\section{Distributed Key Generation}
\label{sec:distributed_key_generation}
Distributed key generation is inspired on secret sharing. The idea behind distributed key generation is that a secret $s$ can be shared among $n$ shareholders without the requirement for a centralised dealer $D$ as in Algorithm~\ref{alg:shamirs_threshold_sheme}. In this way, a secret can be negotiated between all shareholders without any of the shareholders explicitly computing the secret. The major advantage of such a scheme is that no party in the scheme requires a higher level of trust since no party explicitly knows the secret. Similarly to the Shamir secret sharing scheme a group of $t$ or more shareholders will need to pool their shares in order to reconstruct the secret $s$.

\subsection{Definition}
\begin{defn}[Distributed key generation scheme]
\label{def:dkg_scheme}
 A \textit{distributed key generation scheme} is a $\left( t,n \right)$ perfect threshold scheme $\left( t \leq n \right)$ that requires no trusted party. That is, a distributed key generation scheme is a cryptographic scheme that negotiates a secret $s$ with $n$ different parties $P_1, \ldots, P_n$ by letting each party $P_i$ distribute shares $\sigma_{ij}$ of its own private secret $\sigma_i$ with all other parties $P_j$ where $1 \leq i \leq n, 1 \leq j \leq n$. At least $t$ out of $n$ parties will need to collude in order to compute the original secret $s$ explicitly.
\end{defn}

\subsection{Pedersen Distributed Key Generation}
The first usable distributed key generation protocol was defined by Pedersen~\cite{art:Pedersen91a}. A later publication from Gennaro et al.~\cite{art:GennaroJKR07} proves the Pedersen scheme to be insecure in its original form in the presence of malicious key generation centers.

Although the Pedersen scheme~\cite{art:Pedersen91a} is proven insecure, it is most instructive to describe the protocol in its original form as later schemes such as the one from Gennaro et al.~\cite{art:GennaroJKR07} extensively rely on the same concepts. Therefore, the original Pedersen protocol is shown in Algorithm~\ref{alg:pedersen_dkg}. 

The correctness of the Pedersen scheme is based on every party $P_i$ correctly executing a Shamir $\left( t, n \right)$-threshold scheme. After every party has distributed its shares, each party $P_i$ owns $n$ points $\left( i, f_j \left( i \right) \right) = \left( i, \sigma_{ji} \right)$ from other parties $P_j$. By summation of all these shares $s_i = \sum_{j \in \mathcal{Q}} \sigma_{ji} \bmod p$, $s_i$ represents the $y$-coordinate corresponding to $x=i$ of the sum of all curves $f_j \left( x \right)$ generated by all parties $P_j$. In this way, the negotiated master secret key $sk_{msk}$ is equal to the sum of the secret keys of every party $P_i$, such that
 \begin{equation*}
  sk_{msk} = \sum_{i \in \mathcal{Q}} sk_i \bmod p
 \end{equation*}
In fact the master secret key $sk_{msk}$ is found by summing all private curves $f_j \left( x \right)$ such that every honest party's share equally contributes to the master secret key $sk_{msk}$.

The other steps in the Pedersen protocol are required to verify whether all parties calculate the shares honestly. Step e) verifies the correctness of the received shares. Therefore, the $A_{jh}$ values are often called Pedersen commitments. 

The scheme from Gennaro et al.~\cite{art:GennaroJKR07} lets each party $P_i$ commit to two different curves since a flaw in the Pedersen scheme allows malicious parties to determine certain bits in $sk_{msk}$ with non-negligible advantage. The details behind the scheme from Gennaro et al. are omitted since more practical DKG protocols in the asynchronous setting are already presented such as in a publication by Kate and Goldberg~\cite{art:KateHG12}.

\begin{algorithm}
\caption{Pedersen's distributed key generation~\cite{art:Pedersen91a}}
\label{alg:pedersen_dkg}
\begin{description}
 \item \textbf{Goal}: A master secret key $sk_{msk}$ is negotiated with $n$ uniquely numbered parties $\{ P_1, \ldots, P_n \}$ without any of the parties explicitly computing the secret $sk_{msk}$.
 
 \item \textbf{Result}: If a subset of at least $t$ out of $n$ parties colludes, they can reconstruct the original secret $s$.
\end{description}
 \begin{enumerate}
  \item \textit{Setup} At initialisation, a setup algorithm $\left< p, g \right> \leftarrow \mathcal{G} \left( 1^{\lambda} \right)$ is executed that returns a large prime number $p$ and a generator $g$ of $\mathbb{Z}_p$ on input of a security parameter $\lambda$. After execution of $\mathcal{G} \left( 1^{\lambda} \right)$ each party $P_i, 1 \leq i \leq n$ should do the following:
   \begin{enumerate}
    \item $P_i$ generates a random private key $sk_i \in \mathbb{Z}_p$ and publishes the corresponding public key $pk_i = g^{sk_i}$
    \item $P_i$ chooses $t-1$ random independent coefficients $a_{i,1}, \ldots, a_{i,t-1}, 0 \leq a_{i,j} \leq p-1$ defining a random polynomial $f_i \left( x \right)$ over $\mathbb{Z}_p$, $f_i \left( x \right) = \sum^{t-1}_{h=0} a_{i,h} x^h$.
    \item $P_i$ commits to the coefficients $a_{i,1}, \ldots, a_{i,t-1}, 0 \leq a_{i,h} \leq p-1$ by broadcasting $A_{ih} = g^{a_{i,h}} \bmod p$ for $h = 1, \ldots, t$ to all other parties.
    \item $P_i$ computes the share $\sigma_{ij} = f_i \left( j \right) \bmod p$ and securely transfers the share $\sigma_{ij}$ to party $P_j$ along with a signature $S_{P_i} \left( \sigma_{ij} \right)$ authenticating the share. $P_i$ keeps $\sigma_{ii}$ to itself.
    \item $P_i$ verifies for each share $\sigma_{ji}$ received from $P_j$ whether it is consistent by verifying that
    \begin{equation*}
     g^{\sigma_{ji}} = \prod_{h=0}^{n-1} \left( A_{jh} \right)^{i^h} \bmod p
    \end{equation*}
    If the check fails for an index $j$, $P_i$ broadcasts a complaint against $P_j$ along with the received share $\sigma_{ji}$ and its signature $S_{P_j} \left( \sigma_{ij} \right)$. If a party receives $t$ complaints, he is excluded from the set of participating parties $\mathcal{Q}$.
    \item The master public key is calculated as $pk_{msk} = \prod_{j \in \mathcal{Q}} pk_j \bmod p$. The public verification values are calculated as $A_h = \prod_{j \in \mathcal{Q}}A_{jh} \bmod p$ for $k=1, \ldots, t$. Each player $P_i$ sets his share of the secret as $s_i = \sum_{j \in \mathcal{Q}} \sigma_{ji}$.
   \end{enumerate}
   \item \textit{Reconstruction} Any group of $t$ or more shareholders pool their shares. Their shares provide $t$ distinct points $\left( x, y \right) = \left( i, s_i \right)$ allowing computation of the coefficients $a_j, 1 \leq j \leq t-1$ of $f \left( x \right)$ by Lagrange interpolation. The master secret key is recovered by calculating
   \begin{equation*}
    sk_{msk} = \sum_{i \in \mathcal{Q}} s_i b_i \quad \textrm{with} \quad b_i = \prod_{i \in \mathcal{Q}, i \neq h} \frac{h}{h-i}
   \end{equation*}
 \end{enumerate}
\end{algorithm}

\section{Summary}
The aforementioned cryptographic building blocks are all there is required to design a practical encryption scheme for OSNs in Chapter~4.

This chapter started with the introduction of Public Key Infrastructures (PKIs). In a next section, Identity-Based Encryption (IBE) was presented as a possible alternative to traditional PKIs. In addition, the basics of Broadcast Encryption (BE) were highlighted. Finally, secret sharing and Distributed Key Generation (DKG) were introduced as a mechanism to get around the IBE key escrow property.

%%% Local Variables: 
%%% mode: latex
%%% TeX-master: "thesis"
%%% End: 

% ... and so on until
\chapter{Design of a Practical Encryption Scheme for Online Social Networks}
\label{cha:n}
This chapter presents our proposal for practical usable IBE for OSNs. In the first section (Section~\ref{sec:model_of_current_situation}), a model is developed that allows to describe the current OSN situation. By  defining every considered entity in the OSN, the resulting model serves as the framework in which we design our future constructions. In a next step (Section~\ref{sec:threat_model}), the different threats within the model are defined. The section on the threat model first highlights the current privacy threats, followed by a definition of the considered adversaries and the assumptions on these adversaries. In Section~\ref{sec:our_proposal} our proposal is derived by stating cryptographic goals based on the earlier threat model. This is followed by design decisions on how to achieve these goals and how this impacts our model. Section~\ref{sec:our_proposal} is concluded with a concrete proposal in the form of an algorithm along with an evaluation section motivating why our cryptographic design goals are successfully met.

\section{Model of the Current Situation}
\label{sec:model_of_current_situation}
The most commonly accepted definition of an \textit{Online Social Network} (OSN) in literature is from Boyd and Ellison~\cite{art:BoydE08}. However, since this definition is still too generic for the remainder of this text a slightly modified version is presented here.

\begin{defn}[Online Social Network~\cite{art:BoydE08}]
\label{def:osn_boyd}
 An \textit{online social network} (OSN) is a web-based service that allows users to:
 \begin{enumerate}
  \item Construct a public or semi-public profile within a bounded system
  \item Articulate a list of other users with whom they share a connection.
  \item View, traverse and share content with their list of connections and those made by others within the system
  \setcounter{enumTemp}{\theenumi}
 \end{enumerate}
\end{defn}

For the definition of our OSN model we focus on messages sent by users. Therefore, we define several different entities that are present in our model.

\begin{defn}[OSN user]
\label{def:user}
 An \textit{OSN user} $U$ is any entity that has a profile on the OSN and thus identifiable by a unique identifier \id{U}. The set containing all users of an OSN is denoted $\mathcal{U}$.
\end{defn}

An OSN user can perform different activities within the infrastructure of the OSN. Depending on the performed activity, the user is labeled as one of three different roles: a sender, a friend or an intended recipient.

\begin{defn}[Sender]
\label{def:sender}
 A \textit{sender} $A$ is an OSN user who broadcasts a message $m$ over the OSN infrastructure to varying subsets of OSN users, called the \textit{intended recipient set} $\mathcal{S}$, such that $\mathcal{S} \subseteq \mathcal{U}$.
\end{defn}

\begin{defn}[Intended recipient]
\label{def:recipient}
 An \textit{intended recipient} of a plaintext message $m$ is an OSN user who is explicitly designated by a sender $A$ to have access to the content of $m$.
\end{defn}

\begin{defn}[Friend]
\label{def:friend}
 An OSN user who shares a connection with another OSN user $U$ in the OSN infrastructure, is called a \textit{friend of the user $U$}.
\end{defn}

Different entities within the OSN have access rights to the profile $\id{U}$ of a user $U$. If abstraction is made of entities with access to only specific content, it suffices to define four different sets of entities, each with their own access rights: the set of the user's friends $\mathcal{F}_U$, the intended recipient set $\mathcal{S}$, the set of entities with access to the OSN $\mathcal{V}$ and the set of entities with access to the user's profile $\mathcal{V}_U$.

\begin{defn}[Friends set]
\label{def:friends_set}
 The set of all friends associated to a user $U$ is the \textit{friends set} $\mathcal{F}_U$, such that $\mathcal{F}_U \subseteq \mathcal{U}$.
\end{defn}

\begin{defn}[Intended recipient set]
\label{def:intended_recipient_set}
 The \textit{intended recipient set} of a message $m$ is the set of all intended recipients of $m$. The intended recipient set $\mathcal{S}$ takes the form of a list of \id{}'s uniquely identifying other users' profiles in the OSN infrastructure.
\end{defn}

\begin{defn}[Viewers set]
\label{def:viewer}
 Any entity that is given access to the OSN belongs to the \textit{viewers set} $\mathcal{V}$.
\end{defn}

\begin{defn}[Profile viewers set]
\label{def:viewer}
 All viewers with access to non-public content of a profile $\id{U}$ of a user $U$ are in the \textit{profile viewers set} $\mathcal{V}_U \subseteq \mathcal{V}$.
\end{defn}

Many different entities can be part of the set $\mathcal{V}$, e.g. OSN users, advertising companies, system administrators of the OSN or software applications specifically developed for the OSN. Usually, the OSN determines who is part of $\mathcal{V}$. Therefore, a user $U$ often has limited control in who is a member of $\mathcal{V}_U$.

As illustrated by Figure~\ref{fig:current_model}, Sender $U$ wants to broadcast a message $m$ over the OSN infrastructure to the intended recipient set $\mathcal{S}$. As $U$ only wants to share the message with a specific group of friends, $U$ defines the intended recipient set $\mathcal{S}$, such that $\mathcal{S} \subset \mathcal{F}_U$. Next, $U$ sends $m$ to the OSN's distribution server along with the intended recipient set $\mathcal{S}$. The OSN Server further distributes the message to all users in $\mathcal{S}$. Also a subset of third party applications and advertisers get access to the distributed message if they are inside the viewers group $\mathcal{V}_U$. Every entity who has access to the message is coloured blue in Figure~\ref{fig:current_model}.

Figure~\ref{fig:current_model} illustrates previous definitions applied to an OSN as it is often encountered on the internet. The different sets in Figure~\ref{fig:current_model} are defined as follows:
\begin{itemize}
 \item The intended recipient set,
 \begin{equation*}
  \mathcal{S} = \{ \textrm{Recipient}_1, \textrm{Recipient}_2 \}
 \end{equation*}
 \item The set of friends of user $U$,
 \begin{equation*}
  \mathcal{F}_U = \{ \mathcal{S}, \textrm{Friend}_1, \textrm{Friend}_2 \}
 \end{equation*}
 \item The set of viewers who have access to the profile of user $U$,
 \begin{equation*}
  \begin{split}
   \mathcal{V}_U = \{ \mathcal{F}_U, \textrm{Sender } U, \textrm{Advertiser}_1, \textrm{Application}_1 \}
  \end{split}
 \end{equation*}
 \item The set of entities with access to the OSN,
\begin{equation*}
\begin{split}
 \mathcal{V} = \{ \mathcal{V}_U, \textrm{User}_1, \textrm{User}_2, \textrm{Advertiser}_2, \textrm{Application}_2 \}
\end{split}
\end{equation*}
\item The set of all users in the OSN,
\begin{equation*}
 \mathcal{U} = \{ \mathcal{F}_U, \textrm{Sender } U, \textrm{User}_1, \textrm{User}_2\}
\end{equation*}
\end{itemize}


\begin{figure}
    \begin{center}
    \scalebox{0.78}{
        \begin{tikzpicture}[auto, node distance=-2mm, align=center,
            block/.style={rectangle,text width=6em,text centered,minimum height=9mm},
            line/.style={draw,very thick, ->},
            line2/.style={draw,very thick, <->},
            leg/.style={text centered},
            ]
            % Recipient set polygon
            \draw[dashed,color=cyan] (3.5,4) -- (8.5,4) -- (8.5,1.75) -- (3.5,1.75) -- (3.5,4);
            % Friends polygon
            \draw[dashed] (3,4.5) -- (9,4.5) -- (9,-0.25) -- (3,-0.25) -- (3,4.5);
            % Viewers Polygon
            \draw[dashed] (-5,5) -- (9.5,5) -- (9.5,-0.75) -- (-5,-0.75) -- (-5,5);
            % OSN Polygon
            \draw[dashed] (-5.5,5.5) -- (10,5.5) -- (10,-2.75) -- (-5.5,-2.75) -- (-5.5,5.5);
            %\draw[help lines] (-6,-5) grid (8,6);
            \path
                % Images
                (0,3) node [block] (pkg) {\includegraphics[scale=0.15]{img/bluepkg.png}}
                (-4,3) node [block] (alice) {\includegraphics[scale=0.15]{img/bluealice.png}}
                (5,3) node [block] (bob) {\includegraphics[scale=0.15]{img/bluebob.png}}
                (7,3) node [block] (bob1) {\includegraphics[scale=0.15]{img/bluebob.png}}
                (5,1) node [block] (bob2) {\includegraphics[scale=0.15]{img/bob.png}}
                (7,1) node [block] (alice1) {\includegraphics[scale=0.15]{img/alice.png}}
                (-1.5,1) node [block] (adv) {\includegraphics[scale=0.15]{img/bluemoneyman.png}}
                (1.5,1) node [block] (app) {\includegraphics[scale=0.15]{img/blueapp.png}}
                
                (1.5,-1.5) node [block] (app2) {\includegraphics[scale=0.15]{img/app.png}}
                (-1.5,-1.5) node [block] (adv2) {\includegraphics[scale=0.15]{img/moneyman.png}}
                
                
                (5,-1.5) node [block] (alice2) {\includegraphics[scale=0.15]{img/alice.png}}
                (7,-1.5) node [block] (bob3) {\includegraphics[scale=0.15]{img/bob.png}}
                
                (-1.5,-4) node [block] (spy1) {\includegraphics[scale=0.15]{img/spy.png}}
                (1.5,-4) node [block] (spy2) {\includegraphics[scale=0.15]{img/spy.png}}
                % Text
                (0,5.5) node [leg,fill=white] (white_block) {Entities with access to the OSN $\mathcal{V}$}
                (0,5) node [leg,fill=white] (white_block) {Entities with access to $U$'s Profile $\mathcal{V}_U$}
                (6,4) node [color=cyan,leg,fill=white] (white_block) {Intended Recipient Set $\mathcal{S}$}
                (6,4.5) node [leg,fill=white] (white_block) {Friends Set $\mathcal{F}_U$}
                (-2,3.25) node [leg,color=cyan] (white_block) {$m, \mathcal{S}$}
                ;
                
       \node[node distance=2mm, above=of pkg,color=cyan] {\textbf{OSN Broadcast Server}};
       \node[below=of adv,color=cyan] {$\textrm{Advertiser}_1$};
       \node[below=of adv2] {$\textrm{Advertiser}_2$};
       \node[below=of app2] {$\textrm{Application}_2$};
       \node[below=of alice2] {$\textrm{User}_1$};
       \node[below=of bob3] {$\textrm{User}_2$};
       \node[below=of bob,color=cyan] {$\textrm{Recipient}_1$};
       \node[below=of bob1,color=cyan] {$\textrm{Recipient}_2$};
       \node[below=of bob2] {$\textrm{Friend}_1$};
       \node[below=of alice1] {$\textrm{Friend}_2$};
       \node[below=of app2] {$\textrm{Application}_2$};
       \node[below=of alice,color=cyan] {Sender $U$};
       \node[below=of app,color=cyan] {$\textrm{Application}_1$};
       \node[node distance=-6mm,below=of spy1] {$\textrm{Outsider}_1$};
       \node[node distance=-6mm,below=of spy2] {$\textrm{Outsider}_2$};
       \begin{scope}[every path/.style=line,color=cyan]
        \path (alice.east) -- (pkg.west);
        %\path (pkg.south west) -- (adv.north);
        \path (-0.7, 2.3) -- (adv.north);
        \path (0.7,2.3) -- (app.north);
        \path (pkg.east) -- (3.5,3);
       \end{scope}


        \end{tikzpicture}
    }
    \end{center}
    \caption{Model of the current OSN situation. Entities with access to the message $m$ are coloured blue.}
    \label{fig:current_model}
\end{figure}

The OSN's infrastructure stores almost everything within the viewer set $\mathcal{V}$. The profiles of all users within the friends set $\mathcal{F}_B$, the list of \id{}'s within the intended recipient set $\mathcal{S}$, access rights of applications and advertisers that are part of $\mathcal{V}_B$ and access rights of entities within the set $\mathcal{V}$ are all explicitly stored somewhere on the servers of the OSN.

Note that not all OSNs support the functionality to define intended recipient sets $\mathcal{S}$ on a per message basis. In OSNs like Twitter the standard privacy settings are such that message are always published publicly. Therefore, the model from Figure~\ref{fig:current_model} only holds for a specific subset of OSNs like Facebook or Google+. More public OSNs like Twitter would require less sets of entities to model their behaviour.

It requires almost no additional effort to transform the model from Figure~\ref{fig:current_model} such that it also takes the sharing of other media than messages into account. The model could then adopted for use on OSNs like Youtube or Instagram as well. However, this falls out of the scope of this thesis.

\section{Threat Model}
\label{sec:threat_model}
The threat model considers all entities and how they pose a threat in the aforementioned security model.

\subsection{Privacy Threats}
\label{sec:issues_with_current_osn_situation}
Currently an OSN as illustrated in Figure~\ref{fig:current_model} presents several issues that can be classified as either misplaced trust in the OSN provider or dependency on the OSN's privacy infrastructure.

\subsubsection{Misplaced Trust in the OSN Provider}
Users have to trust the OSN provider, however this trust is often misplaced on several fronts.

\paragraph{Trust in the intended recipient set} There is a mismatch between the expectations of Sender $U$ and the functionality of the OSN. A user has higher demands on the specification of the intended recipient set than effectively implemented by the OSN. When a privacy-aware user like sender $U$ takes the effort to define an intended recipient set $\mathcal{S}$, $U$ expects to have full control on who has access to the messages $m$. In reality, sender $U$ only has partial control since the OSN determines all other entities in $\mathcal{V}_U$ that are not part of sender $U$'s friend list $\mathcal{F}_U$. In some OSNs a user first has to give permission to third party applications before access is granted to the user's content. Note that this gives more control to OSN users on determining who is inside the viewers set $\mathcal{V}_U$. Nevertheless, in practice it is still hard to get a concise overview from the OSN on everyone inside $\mathcal{V}_U$.

\paragraph{Trust in the broadcasting mechanism} Any user broadcasting messages over the OSN infrastructure has to trust the OSN that it effectively operates as claimed. If the OSN broadcast server in Figure~\ref{fig:current_model} would accidentally broadcast messages publicly despite of the sender's privacy settings, the privacy of the sender is breached. Even more worrisome is the fact that the sender will almost certainly never discover these privacy violations.

\paragraph{Trust in the data storage policy} All messages are stored on the infrastructure provided by the OSN. However, has to trust the OSN will treat this data responsible. Nevertheless, most OSN data storage policies store all content for an unlimited amount of time on their servers. The user thereby loses the control over his own data and to whom and how long it is still available.

\paragraph{Mismatch of Needs} Besides the earlier mentioned issues, the OSN often operates with a corporate mentality. Although the user desires a relatively private OSN environment, the needs of the OSN are different. The OSN has no initiative to stop adding advertisers and applications to the set of entities with access to a user's profile $\mathcal{V}_B$. The more information advertising companies receive from the OSN provider, the better they can tailor advertisements to the user. The more third party applications rely on the OSNs infrastructure, the more appealing the OSN business model looks like. Therefore, OSNs have often not enough incentives to offer stricter access control policies to their users.

\subsubsection{Dependency on the OSN's Privacy Infrastructure}
Often, users are also dependent on the privacy infrastructure as it is currently provided by the OSN.

\paragraph{Dependency on accessibility of infrastructure}
Users of the OSN have to rely on the security of the OSN's infrastructure. If one of the outsiders in Figure~\ref{fig:current_model} would succeed in hacking the OSN's digital infrastructure, he would have immediate access to all sensible information stored on the OSN's servers. Similarly, local governments can subpoena the OSN to disclose sensible information on certain users with the argument of national security.

\paragraph{Dependency on privacy preferences} Another significant point is that the OSN fully determines which access policies are supported. Not all OSNs offer the definition of an intended recipient set on a per message basis. Even OSNs currently supporting this functionality can suddenly stop offering the service. Moreover, nothing prevents OSN providers from changing their privacy policy on a regular basis, thereby complicating users to define the access policy of their choice.


\subsection{Adversaries}
We consider adversaries that are honest but curious, i.e. passive adversaries that do not actively try to prevent the broadcasting process but are curious for the broadcasted content. An adversary in the earlier defined model, is any computationally bounded entity trying to violate one or several of the following properties:
\begin{enumerate}
 \item \textbf{Confidentiality:} The entity tries to violate the confidentiality of encrypted messages, i.e. uncovering information within a broadcasted ciphertext. This can either be the intended recipient set $\mathcal{S}$ or the actual content of the plaintext message $m$.
 \item \textbf{Integrity:} The entity tries to violate the integrity of encrypted messages, i.e. changing the ciphertext $c$ or the plaintext $m$ such that it differs from the way it was originally drafted by the sender.
 \item \textbf{Availability:} Any entity apart from the original sender, tries to prevent messages from being broadcasted by bringing down parts of the architecture or the OSN. 
\end{enumerate}

\subsection{Assumptions}
The threat model can narrow down the capabilities of entities in the model to achieve well-defined adversaries. However, an architecture protecting against more restricted adversaries is considered less secure.

\subsubsection{Assumptions on the OSN}
OSNs are assumed not to violate integrity neither availability. Nothing can be done to prevent the OSN from actively altering its own resources to bring down the proposed IBE architecture. It can not be prevented that a user of IBE on the OSN infrastructure gets blocked by the OSN provider. Neither can it be prevented that OSNs delete messages because they are encrypted. OSNs can also easily impersonate the owner of a profile in their infrastructure. Therefore, from this moment onwards, OSNs are assumed not to act as an active adversary. It is assumed that as soon as privacy aware users notice this kind of OSN behaviour, they will adopt to more reliable OSN alternatives. However, our assumptions do not prevent the OSN from trying to passively break confidentiality as they have every motivation for it in the context of their current business model.

Another assumption on the OSN's infrastructure is that the authentication mechanism of the OSN is secure. This is primarily important for the authenticity of the broadcasted messages on a user's profile as further discussed in Section~\ref{sec:authenticity_and_integrity}.

Previous assumptions on the OSN are ideal assumptions that will probably hold as long as only the minority of the OSN users applies encryption mechanisms to their network. It is still unknown how OSNs will react if the proposed encryption mechanism finds common acceptance in their wide user base.

Furthermore, it is assumed that the OSN does not rely on traffic analysis to derive more information on the encrypted content. For example, suppose a user just visited a URL to an external news website broadcasted over the OSN infrastructure. With high probability the message broadcasted immediately after the user's visit, is a reaction to the content in the article. Furthermore, if Bob visits Alice's profile on a daily basis, almost certainly Bob regularly includes Alice as a recipient in his broadcasted messages as well.

Note that the latter assumption is not valid in practice. Generally, the attractiveness of traffic analysis is tempered by generating dummy traffic. However, protecting against traffic analysis falls out of the scope of this thesis.

\subsubsection{Assumptions on the User}

In order to achieve a strong encryption mechanism, users are unlimited in their abilities to behave as an adversary. However, one subtle assumption is made on users that are part of $\mathcal{S}$. Users in $\mathcal{S}$ are assumed not to break their social contract. That is, if a sender broadcasts a confidential message to a selected set of recipients, the recipients are assumed not to decrypt the encrypted message and rebroadcast the confidential content to any entity not in the original recipient set. In fact, no existing encryption mechanism provides protection against such misbehaviour, although traitor tracing schemes~\cite{art:ChorFNP00} discourage users from treating by indicating who broke his social contract. However, for the remainder of this text intended recipients are assumed trustworthy in that they do not break the social contract.

\section{Our Proposal}
\label{sec:our_proposal}
Our well-defined security model is used as a framework to define a concrete proposal protecting against the earlier defined adversaries.

\subsection{Cryptographic Goals}
\label{sec:cryptographic_goals}
The general design goals (Section~\ref{sec:goals_of_this_thesis}) stated that a new privacy enhancing architecture for OSNs should be user friendly, applicable and immediately ready to use. Besides from these general design goals, it is now possible to define specific cryptographic requirements as well. A well-designed encryption scheme should be able to achieve the following cryptographic goals when publishing a message $m$ to a set of intended recipients $\mathcal{S}$ on an OSN with the help of an encryption scheme:
\begin{itemize}
 \item \textbf{Confidentiality:} The message is protected from disclosure to unauthorised parties, i.e. all entities that are not explicitly in the recipient set $\mathcal{S}$.
 \item \textbf{Outsider recipient anonymity:} The intended recipients of a broadcasted message should be anonymous to any entity not included in $\mathcal{S}$. This implies that neither the OSN has to know who the recipients are. (Definition~\ref{def:outsider_anonymity} gives a more formal definition of outsider-anonymity).
 \item \textbf{No redundancy:} The message should be published only once to reach every recipient in $\mathcal{S}$.
 \item \textbf{Authenticity:} The recipients of the message have reasonable assurances of the message's origin.
 \item \textbf{Integrity:} The recipients are assured the message is distributed in its original form as posted by the sender.
 \item \textbf{No key escrow:} Private keys are only disclosed to the owners of the public key. No other entity should be able to have more information on one's secret key in the information theoretic sense.
 \item \textbf{Key validation:} All users of the system should be able to verify the correctness of their private keys.
 \item \textbf{Limited key validity:} Private keys of users should only be valid for a limited period of time to limit the damage of potentially lost private keys.
\end{itemize}

\subsection{Design Decisions}
\label{sec:design_decisions}
In this section, our architecture is further developed by keeping the aforementioned cryptographic design goals in mind.

\subsubsection{Confidentiality}
Confidentiality can be achieved by applying an encryption scheme before broadcasting a message. Current solutions like Scramble~\cite{art:BeatoKW11} and Persona~\cite{art:BadenBSBS09} rely on rather classic public key infrastructures thereby requiring the OSN user to subscribe to a third party key infrastructure. These key infrastructures are required to authenticate and store the public keys of all security aware users. However, this does not correspond to the general design goals from Section~\ref{sec:goals_of_this_thesis} stating that the proposed solution should be both user friendly and immediately ready to use.

Identity-based encryption (IBE) can be used to achieve both confidentiality and the general design goals of usability and applicability. During the design of our scheme, three IBE schemes were considered as a potential candidate: Boneh and Franklin IBE~\cite{art:BonehF01}, Sakai and Kasahara IBE~\cite{art:SakaiK03} and Gentry IBE~\cite{art:Gentry06}. For a more elaborate discussion on why only these schemes were considered, the reader is referred to Section~\ref{sec:evolution_of_be}.

Table~\ref{tab:ibe_security_comparison} lists the different security properties of all schemes. The Gentry IBE scheme has the highest security level since it is the only scheme proven secure in the standard model. In the random oracle model, Boneh and Franklin IBE is preferred over Sakai and Kasahara IBE since it relies on the BDH assumption which is more widely accepted than the stronger BDHI assumption.

The execution times of all considered IBE schemes are illustrated in Table~\ref{tab:ibe_performance_comparison}. We conducted our measurements on an Intel Core 2.4 GHz i5 processor with 8 Gb of 1600 MHz DDR3L onboard memory. Pairing computations were implemented using the multi-precision MIRACL library~\cite{art:Scott03}. The Gentry IBE scheme was first transformed to the asymmetric setting to give a fair basis of comparison. The exact transformed Gentry IBE scheme is depicted in Appendix~\ref{app:gentrys_ibe_scheme}. 

Table~\ref{tab:ibe_performance_comparison} clearly illustrates the price there is to pay for security in the standard model. Therefore, Boneh and Franklin IBE was chosen as the preferred IBE scheme despite the dependency on the random oracle assumption.


\begin{table}
  \centering
  \begin{tabular}{@{}lccr@{}} \toprule
    \multicolumn{3}{r}{Security Proof} \\ \cmidrule(r){2-4}
    IBE Scheme    & IND-ANO-CCA & Standard model & Assumption \\ \midrule
    Boneh and Franklin & \Checkmark & \XSolidBrush  & BDH \\
    Sakai and Kasahara & \Checkmark & \XSolidBrush & BDHI \\
    Gentry & \Checkmark & \Checkmark & q-BDHE \\ \bottomrule
  \end{tabular}
  \caption{Security comparison of considered IBE schemes}
  \label{tab:ibe_security_comparison}
\end{table}

\begin{table}
  \centering
  \begin{tabular}{@{}lrrrr@{}} \toprule
    \multicolumn{4}{r}{Execution time (ms)} \\ \cmidrule(r){2-5}
    IBE Scheme    & IBE.Setup & IBE.Extract & IBE.Encrypt & IBE.Decrypt \\ \midrule
    Boneh and Franklin & 368.10 & 13.84 & 271.90 & 252.82 \\
    Sakai and Kasahara & 1257.72 & 20.49 & 319.83 & 259.17\\
    Gentry & 24.49 & 37.46 & 1136.65 & 911.32 \\ \bottomrule
  \end{tabular}
  \caption{Performance comparison of considered IBE schemes in MIRACL}
  \label{tab:ibe_performance_comparison}
\end{table}

IBE requires that OSN profiles can be uniquely identified by a unique public identifier \id{}. However, the decision on which string to use as identifier is highly dependent on the underlying OSN and therefore implementation dependent.

\subsubsection{Outsider Recipient Anonymity}
The outsider anonymity requirement from Section~\ref{sec:cryptographic_goals} is imposed on the recipient set since our solution is developed in the context of OSNs where user interaction plays an important role. Therefore, it is useful that members of the intended recipient set $\mathcal{S}$ know each other. For example, suppose that Alice broadcasts an encrypted message intended to Bob and Dylan using a scheme that fully hides the identity of the recipients. This implies that $\id{Bob}, \id{Dylan} \in \mathcal{S}$. As a reaction to Alice's message, Bob wants to write a reply to start a discussion. However, as Bob does not know which other users are allowed to see Alice's message, he can now only encrypt his reply to Alice thereby preventing Dylan from joining the discussion. Nevertheless, this discussion could have been useful to Dylan as well because otherwise Alice would not have included Dylan as a recipient in $\mathcal{S}$ in the first place.

From the outsider-anonymity requirement, it immediately follows that users not necessarily need to be friends to receive each other's messages. In the specific example of Alice, Bob and Dylan, it could be that Bob and Dylan both have Alice as a common friend while no immediate friend connection exists between Bob and Dylan. This should be taken into consideration when determining the identifiers of Bob's and Dylan's profiles, $\id{Bob}$ and $\id{Dylan}$ respectively.

As discussed in Section~\ref{sec:anobe}, broadcast encryption schemes can be made more efficient if the recipient set $\mathcal{S}$ is public. So if user interaction is really that important, why not make the intended recipient set public? Consider the example in which Bob's girlfriend celebrates her birthday in a few weeks. When Bob's girlfriend notices that Bob broadcasted an encrypted message to all her friends without including her as a recipient, she will probably know Bob is up to something. This is just one specific example that illustrates the negative impact on security, broadcasting of the recipient set $\mathcal{S}$ can have on real life situations. Depending on the context, information can be deduced about the message without decrypting it to plain text.

\subsubsection{No redundancy}
From the no redundancy requirement it immediately follows that a broadcast encryption scheme should be used, preferably one that hides the anonymity of recipients in the intended recipient set $\mathcal{S}$ to the outside world. However, apart from the outsider-anonymous broadcast encryption scheme from Fazio and Perera~\cite{art:FazioP12}, no efficient schemes of this kind are described in literature. Since the BE scheme from Fazio and Perera does not fully benefit from the advantages of IBE, the ANOBE scheme from Libert et al.~\cite{art:LibertPQ12} is preferred for further implementation. Since recipients still have to know who else is included in $\mathcal{S}$, the list of \id{}s within $\mathcal{S}$ is concatenated to the plaintext message before encryption.

The scheme from Libert et al. also offers non-repudiation by using signature schemes. Note however, that a trusted authority authorising and publishing the public keys is required for the implementation of signature schemes. Because the general design goals were applicability and user friendliness, no third party PKI can be supported. Therefore, the implemented scheme does not rely on signatures as in the original proposal from Libert et al.~\cite{art:LibertPQ12}.

If the security parameter is chosen to be $\lambda$, the IBE scheme in Algorithm~\ref{alg:full_indent} can only encrypt messages with a maximum length of $l$ bits. This can be seen since in the last step of \texttt{IBE.Encrypt} the message $m$ is encrypted by an XOR operation with the result of a hash function $H_3: \{ 0,1 \}^l \rightarrow \{ 0,1 \}^l$. Because asymmetric IBE schemes can only encrypt these fixed length messages, the scheme from Libert et al.~\cite{art:LibertPQ12} is altered such that the ciphertext in the original proposal contains a with IBE encrypted symmetric session key $k$ that is the same for each user in the recipient set $\mathcal{S}$ on a per message basis. The actual plaintext is then encrypted with a symmetric encryption scheme based on a mode of operation to support longer message lengths.

\subsubsection{Authenticity and Integrity}
\label{sec:authenticity_and_integrity}
Authenticity and integrity can be achieved at the same time by relying on an authenticated encryption scheme. The integrity of a message is then as strong as the security guarantees of the authenticated encryption scheme. 

Note however, that the authentication mechanism still relies on the security guarantees of the OSN. Since no third party PKI mechanism is used, there is no trusted party verifying the identity corresponding to a public key. In OSNs this is not an issue if IBE is used with unique profile identifiers as a public key. Consequently, such an IBE scheme ensures that messages encrypted under a public identifier can only be seen by the owner of the corresponding OSN profile. Verifying whoever owns the OSN profile remains the responsibility of the OSN and the judgement of the OSN profile's connections. However, if the authentication mechanism of the OSN is inadequate, anyone could login to a user's profile to impersonate the actual owner of the profile. Therefore, our proposed solution can not be more secure than the authentication mechanism of the OSN.

In more traditional communication schemes, authenticated encryption uses the symmetric key as agreed during an authenticated key agreement protocol like the Station-to-Station protocol~\cite{art:DiffieOW92}. Authenticity of ciphertexts generated by the authenticated encryption scheme than immediately follows from the usage of the same symmetric session key $k$ as earlier agreed during the protocol. However, since in the proposed solution every OSN user should be able to immediately broadcast confidential messages to other users of the OSN, no key agreement protocols will be used. With the publication of only one broadcast ciphertext, every user in the intended recipient set $\mathcal{S}$ should be immediately able to decrypt it to the original plaintext message $m$. Therefore, there is no real authenticity in the value of the tag $t$ generated by the authenticated encryption scheme because anyone with access to the user's profile could have chosen a random symmetric session key $k$ and have used it as an input to the authenticated encryption scheme. Unless, the only one with access to the user's profile is the actual owner of the profile. Therefore, the authenticity guaranteed by the authenticated encryption scheme boils down to the security of the authentication mechanism as powered by the OSN.


\subsubsection{No Key Escrow and Key Validation}
One of the major drawbacks of IBE schemes is that they inherently imply key escrow (Section~\ref{sec:pros_and_cons_of_ibe}). To circumvent the key escrow property of IBE schemes, multiple PKGs can be used implementing a distributed key generation (DKG) mechanism for IBE. Users can then verify their private keys by relying on the basics of commitment schemes (Definition~\ref{def:commitment_scheme}).

For the exact details on how a commitment scheme can achieve this verification mechanism, the reader is immediately referred to the exact proposed scheme in Section~\ref{sec:proposed_scheme}.

\subsubsection{Limited Key Validity}
IBE schemes do not allow revocation of public keys (Section~\ref{sec:pros_and_cons_of_ibe}). A solution to circumvent this drawback is by concatenating an expiration date to all public identifiers \id{}. However, these expiration dates should be publicly available to all OSN users since they are part of the public IBE key. To avoid the management of a third party infrastructure keeping track of expiration dates of all users, a special type of function could be used mapping identifiers \id{} to dates. An example of such a function is shown in Algorithm~\ref{alg:map_to_date}.

Algorithm~\ref{alg:map_to_date} is constructed such that step 1 to 4 only need to be executed once. The sender then stores values $d_1, h_1, m_1$ locally and only repeats step 5 for each recipient of the message. The exact implementation details could be hidden from the user in software. Different variants of Algorithm~\ref{alg:map_to_date} could be applied as well. The most important aspect is that everyone in the system uses the same function to map strings to expiration dates.

\begin{algorithm}
\caption{A function mapping strings to dates}
\label{alg:map_to_date}
\begin{description}
 \item \textbf{Goal}: Avoid a third party infrastructure that keeps track of expiration dates of key pairs in an IBE system
 
 \item \textbf{Result}: On input of a public identifier \id{} the algorithm returns an expiration date in the form \texttt{d/M/y h:m}.
\end{description}
 \begin{enumerate}
  \item Choose a hash function $H: \{ 0, 1 \}^* \rightarrow \{ 0,1 \}^l$ mapping binary strings of arbitrary length to binary strings of a fixed length $l$.
  \item Calculate $r = H \left( \id{} \right)$ and interpret the result $r$ as an integer.
  \item Calculate $tot_m = r \bmod 40320 = r \bmod \left( 60 \cdot 24 \cdot 28 \right)$, where $tot_m$ denotes the expiration time in minutes within a certain month.
  \item Calculate three integers $d_1, h_1, m_1$ denoting an expiration day, hour and minute respectively with $1 \leq d_1 \leq 28$, $0 \leq h_1 \leq 23, 0 \leq m_1 \leq 59$ as follows 
  \begin{enumerate}
   \item The expiration minute is calculated as $m_1 = tot_m \bmod 60$
   \item The expiration hour is calculated as $h_1 = \frac{tot_m}{60} \bmod 24$
   \item The expiration day is calculated as $d_1 = \frac{tot_m}{60 \cdot 24}$
  \end{enumerate}
  It can be shown that $d_1, h_1, m_1$ are chosen uniformly random within their boundaries if the random oracle assumption holds for the hash function $H \left( \cdot \right)$.
  \item Let \texttt{nowIsEarlierThan($d_1$,$h_1$,$m_1$)} be a function that returns \texttt{true} if the current time \texttt{d/M/y h:m} is before \texttt{$d_1$/M/y $h_1$:$m_1$} and \texttt{false} otherwise. Output the expiration date as
  \begin{enumerate}
   \item If \texttt{nowIsEarlierThan($d_1$,$h_1$,$m_1$)} = \texttt{true}, return \texttt{$d_1$/M/y $h_1$:$m_1$}.
   \item If \texttt{nowIsEarlierThan($d_1$,$h_1$,$m_1$)} = \texttt{false} and \texttt{M}$+1 \leq 12$, return \texttt{$d_1$/M/y $h_1$:$m_1$}.
   \item Else return \texttt{$d_1$/1/(y+1) $h_1$:$m_1$}
  \end{enumerate}
 \end{enumerate}
\end{algorithm}


\subsection{Updated Model}
\label{sec:updated_model}
For the sake of completeness, PKGs are added as an additional entity to our model. 

\begin{defn}[PKG in our security model]
\label{def:dkg}
 A \textit{Public Key Generator} (PKG) is an entity in the security model that never colludes with any other entity in the model since its prime motivation is to improve the current security situation in OSNs.
\end{defn}

\begin{figure}[ht]
    \begin{center}
    \scalebox{0.78}{
        \begin{tikzpicture}[auto, node distance=-2mm, align=center,
            block/.style={rectangle,text width=6em,text centered,minimum height=9mm},
            line/.style={draw,very thick, ->},
            line2/.style={draw,very thick, <->},
            leg/.style={text centered},
            ]
            % Recipient set polygon
            \draw [color=cyan] (3.5,4) -- (8.5,4) -- (8.5,2) -- (3.5,2) -- (3.5,4);
            % Friends polygon
            \draw[dashed] (3,4.5) -- (9,4.5) -- (9,-0.25) -- (3,-0.25) -- (3,4.5);
            % Viewers Polygon
            \draw[dashed] (-5,5) -- (9.5,5) -- (9.5,-0.75) -- (-5,-0.75) -- (-5,5);
            % OSN Polygon
            \draw[dashed] (-5.5,5.5) -- (10,5.5) -- (10,-2.75) -- (-5.5,-2.75) -- (-5.5,5.5);
            %\draw[help lines] (-6,-5) grid (8,6);
            \path
                % Images
                (0,3) node [block] (pkg) {\includegraphics[scale=0.15]{img/pkg.png}}
                (-4,3) node [block] (alice) {\includegraphics[scale=0.15]{img/bluealice.png}}
                (5,3.25) node [block] (bob) {\includegraphics[scale=0.15]{img/bluebob.png}}
                (7,3.25) node [block] (bob1) {\includegraphics[scale=0.15]{img/bluebob.png}}
                (5,1) node [block] (bob2) {\includegraphics[scale=0.15]{img/bob.png}}
                (7,1) node [block] (alice1) {\includegraphics[scale=0.15]{img/alice.png}}
                (-1.5,1) node [block] (adv) {\includegraphics[scale=0.15]{img/moneyman.png}}
                (1.5,1) node [block] (app) {\includegraphics[scale=0.15]{img/app.png}}
                
                (1.5,-1.5) node [block] (app2) {\includegraphics[scale=0.15]{img/app.png}}
                (-1.5,-1.5) node [block] (adv2) {\includegraphics[scale=0.15]{img/moneyman.png}}
                
                % DKGs
                (-4,7) node [block] (pkg1) {\includegraphics[scale=0.15]{img/pkg.png}}
                (0,7) node [block] (pkg2) {\includegraphics[scale=0.15]{img/pkg.png}}
                (4,7) node [block] (pkg3) {\includegraphics[scale=0.15]{img/pkg.png}}
                (8,7) node [block] (pkg4) {\includegraphics[scale=0.15]{img/pkg.png}}
                
                (5,-1.5) node [block] (alice2) {\includegraphics[scale=0.15]{img/alice.png}}
                (7,-1.5) node [block] (bob3) {\includegraphics[scale=0.15]{img/bob.png}}
                
                (-1.5,-4) node [block] (spy1) {\includegraphics[scale=0.15]{img/spy.png}}
                (1.5,-4) node [block] (spy2) {\includegraphics[scale=0.15]{img/spy.png}}
                % Text
                (0,5.5) node [leg,fill=white] (white_block) {Entities with access to the OSN $\mathcal{V}$}
                (0,5) node [leg,fill=white] (white_block) {Entities with access to $B$'s Profile $\mathcal{V}_B$}
                (6,2) node [leg,fill=white] (white_block) {Intended Recipient Set $\mathcal{S}$}
                (6,-0.25) node [leg,fill=white] (white_block) {Friends Set $\mathcal{F}_B$}
                (-2,3.25) node [leg] (white_block) {$c$}
                ;
                
       \node[node distance=2mm, above=of pkg] {\textbf{OSN Broadcast Server}};
       \node[below=of adv] {Advertiser 1};
       \node[below=of adv2] {Advertiser 2};
       \node[below=of app2] {Application 2};
       \node[below=of alice2] {User 1};
       \node[below=of bob3] {User 2};
       \node[below=of bob,color=cyan] {Recipient 1};
       \node[below=of bob1,color=cyan] {Recipient 2};
       \node[below=of bob2] {Friend 1};
       \node[below=of alice1] {Friend 2};
       \node[below=of app2] {Application 2};
       \node[below=of alice,color=cyan] {Sender $A$};
       \node[below=of app] {Application 1};
       \node[below=of pkg1] {\textbf{PKG 1}};
       \node[below=of pkg2] {\textbf{PKG 2}};
       \node[below=of pkg3] {\textbf{PKG 3}};
       \node[below=of pkg4] {\textbf{PKG 4}};
       \node[node distance=-6mm,below=of spy1] {Outsider 1};
       \node[node distance=-6mm,below=of spy2] {Outsider 2};
       \begin{scope}[every path/.style=line]
        \path (alice.east) -- (pkg.west);
        %\path (pkg.south west) -- (adv.north);
        \path (pkg.east) -- (3,3);
        \path (-0.7, 2.3) -- (adv.north);
        \path (0.7,2.3) -- (app.north);
       \end{scope}
       \begin{scope}[every path/.style=line2]
        \path (4,6) -- (bob);
        \path (7.75,6) -- (bob);
        \path (8,6) -- (bob1);
        \path (4.25,6) -- (bob1);
       \end{scope}


        \end{tikzpicture}
    }
    \end{center}
    \caption{Model of the desired OSN situation. Entities with the ability to decrypt the ciphertext are coloured blue.}
    \label{fig:new_model}
\end{figure}

In the threat model, PKGs are considered to always behave as described in the DKG protocol. Note that this is a simplification of a PKG as it is often encountered in real-world applications. However, considering PKGs as malicious requires far more complex distributed key generation algorithms which are out of the scope of this thesis. For DKG protocols that can be used in more hostile PKG environments as encountered in practice, the reader is referred to Kate et al.~\cite{art:KateHG12}.

Figure~\ref{fig:new_model} illustrates the changes on the original model of the OSN situation in Figure~\ref{fig:current_model}. At the top of Figure~\ref{fig:new_model} four PKGs are introduced implementing a $\left( t, n \right)$ DKG protocol with $t = 2$ and $n = 4$ (Section~\ref{sec:distributed_key_generation}). Double arrows represent the secure authentication process in which the recipients communicate with the PKG to receive a share of their secret key. In Figure~\ref{fig:generic_ibe_scheme} this communication was illustrated by two separate single arrows between the PKG and Bob. However, abstraction is made of the exact communication protocol between the recipients and the PKG.

Apart from the newly introduced PKGs, Figure~\ref{fig:new_model} differs from Figure~\ref{fig:current_model} in several ways. Sender $A$ no longer specifies the set of intended recipients $\mathcal{S}$ to the OSN broadcast server. Therefore, the OSN broadcast server delivers the message to all entities with access to $A$'s profile $\mathcal{V}_A$. Note that even Friend 1 and 2 are able to see the broadcasted message which was not the case in Figure~\ref{fig:current_model}. However, since the broadcasted message is actually a ciphertext $c$ of the original message $m$, only the entities in blue will be able to read the confidential content of the original plaintext message $m$.

%Due to the proposed encryption mechanisms, the OSN provider storing the broadcasted ciphertext messages should not be able to derive more information than any other entity in the model. However, careless implementation of the proposed encryption mechanism could still give away more information to the OSN than desired.

%A sender should never use the privacy mechanisms provided by the OSN. If a sender $B$ specifies the intended recipient set $\mathcal{S}$ in the infrastructure of the OSN, the recipient set is obviously no longer private. In this case, both the OSN as well as other users within the set of entities with access to the senders profile $\mathcal{V}_B$ can possibly learn information on the recipients and the content of the message.




%%%%%%%%%%%%%%%%%%%%%%%%%%%%%%%%%%%%%
%% \subsection{Active Adversaries} %%
%%%%%%%%%%%%%%%%%%%%%%%%%%%%%%%%%%%%%

%Active adversaries try to alter system resources to affect their operation or actively take part in the protocol to derive more information from the secret data than actually is allowed.
% Eén DKG neerhalen is onvoldoende vermits threshold DKG wordt gebruikt
% Gebruikers kunnen hun private sleutel kwijt geraken

\subsection{Scheme}
\label{sec:proposed_scheme}
Taking all aforementioned cryptographic design decisions into account results in the scheme presented in Algorithm~\ref{alg:our_scheme}.

\subsection{Evaluation}
Algorithm~\ref{alg:our_scheme} achieves the earlier stated cryptographic design goals from Section~\ref{sec:cryptographic_goals}. However, it is too early to conclude on the more general design goals from the introduction (Section~\ref{sec:goals_of_this_thesis}) since the achievement of this goals is implementation dependent. The cryptographic goals are realised as follows:

\paragraph{Confidentiality} The proposed scheme achieves confidentiality as in~\cite{art:BonehF01,art:LibertPQ12}, because a session key $k$ can only be obtained if the recipient holds the corresponding secret key $s_{\id{i}}$ to an identifier $\id{i}$ that is included in $\mathcal{S}$. Confidentiality of $k$ is thus guaranteed by ANO-IND-CCA secure Boneh and Franklin IBE~\cite{art:BonehF01}. Confidentiality of the plaintext message $m$ than immediately follows from the confidentiality of the authenticated encryption in step 5 of \texttt{Publish}.

\paragraph{Outsider recipient anonymity} Our solution is made anonymous by relying on the ANO-IBE scheme from Boneh and Franklin~\cite{art:BonehB04}. Furthermore, the broadcasting mechanism is inspired by the BE scheme from Libert et al.~\cite{art:LibertPQ12} which is recipient anonymous as well. The BE scheme is applied to broadcast $k$. In terms of efficiency, users are required to decrypt $w_i$ on average $O\left( \eta /2 \right)$ before obtaining $k$ due to the anonymity of the BE scheme. Both Barth et al.~\cite{art:BarthBW06} and Libert et al.~\cite{art:LibertPQ12} propose using a tag based system to hint users where they can find their symmetric key. However, it was deliberately decided not to implement such property in the scheme as it introduces a dependency of the public parameters linear in the total number of users in the system.

Algorithm~\ref{alg:our_scheme} is outsider recipient anonymous due to the concatenation of the recipient set to the plaintext message $m$ in step 4 of \texttt{Publish}.

\paragraph{No redundancy} The broadcast message $\mathcal{B}$ should only be published once since $\mathcal{B}$ contains a concatenation $w$ of the with IBE encrypted session key $k$ for every recipient in $\mathcal{S}$.

\paragraph{Authenticity and integrity} Authenticity and integrity are guaranteed by the authenticated symmetric encryption scheme and the authentication mechanism of the OSN. If the assumptions on the OSN's authentication mechanism hold, only the owner of an OSN profile should be able to actively broadcast messages in name of the corresponding profile identifier $\id{i}$.

%%%%%%%%%%%%%%%%%%%%%
% Hacky Latex stuff %
%%%%%%%%%%%%%%%%%%%%%
%%%%%%%%%%%%%%%%%%%%%%%%%%%%%%%%%%%%%
% TODO: make sure the algorithm fits on two consecutive pages
%%%%%%%%%%%%%%%%%%%%%%%%%%%%%%%%%%%%%
\newpage

\thispagestyle{empty}

\makeatletter
\setlength{\headsep}{-10pt}
\makeatother

\begin{algorithm}[H]
\caption{An outsider recipient anonymous identity-based broadcast encryption scheme}
\label{alg:our_scheme}

\begin{description}
    \item[\texttt{Setup($\lambda, t, n$)}:] Outputs the public $params$ of the system with respect to the security parameter $\lambda$, the number of PKGs $n$ and the threshold $t$.
    \begin{enumerate}
        \item On input of security parameter $\lambda$ generate a prime $q$, two groups $G_1, G_2$ of order $q$, and an admissible bilinear map $e: G_1 \times G_2 \rightarrow G_T$. Choose random generators $P \in G_1$ and $Q \in G_2$. 
    
        \item Choose cryptographic hash functions $H_1: \{ 0,1 \}^{*} \rightarrow G_1$, ${H_2: G_T \rightarrow \{ 0,1 \}^{l}}$ and $H_3: \{ 0, 1 \}^{l} \rightarrow \{ 0,1 \}^{l}$, such that $H_1, H_2$ can be modelled as random oracles.
        
        \item Each PKG $j$ generates $n-1$ shares $\sigma_{jv}$ of a Pedersen VSS scheme by executing \texttt{DKG.Setup}, and redistributing the $n-1$ shares $\sigma_{jv}$ with the other $v$ PKGs.

        \item Each PKG $j$ publishes $P_{pub}^{(j)} = s_j P$, s.t., $s_j=\sum_{v=1}^n \sigma_{jv}$.
    \end{enumerate}
    
    The master secret key $sk_{msk} = \sum_{j \in \Lambda} b_j s_j$ for $b_j = \prod_{z \in \Lambda} \frac{z}{z-j}$ cannot be retrieved unless $\Lambda$ is a subset of size $t$ different PKG servers. The following parameters are published publicly:
    \begin{equation*}
    params = \{ q, G_1, G_2, e, P, Q, H_1, H_2, H_3, t, n, P_{pub}^{(0)}, \ldots, P_{pub}^{(n)} \}
    \end{equation*}

    \item[\texttt{KeyGen(\{PKG$_0,\ldots,$PKG$_t\}, \id{i}$)}:] On input of a user $\id{i}$ the subset $\Lambda$ of size $t$ of PKG servers, generates a valid private key for \id{i}. 
    
    \begin{enumerate}
        \item User with identifier \id{i}, authenticates to $\Lambda$ or all PKGs and sends \id{i}.
        \item Each PKG computes $Q_{\id{i}} = H_1 \left( \id{i} \right)$, and $Q_{priv,\id{i}}^{(j)} = s_j Q_{\id{i}}$, where $s_j$ is the secret share from PKG $j$.
        \item The user $\id{i}$ computes the shared public parameter $P$ using the Lagrange coefficients $b_j$ as follows:
        \begin{equation*}
         P = \sum_{j \in \Lambda} b_j P_{pub}^{\left( j \right)} \quad \textrm{for} \quad b_j = \prod_{z \in \Lambda} \frac{z}{z-j}
        \end{equation*}
        \item All PKGs in $\Lambda$ return $Q_{priv,\id{i}}^{(j)}$ to the corresponding user $\id{i}$ over a secure channel.
        \item Each user verifies for each $Q_{priv,\id{i}}^{(j)}$ value whether, 
        \begin{equation*}
            e \left( Q_{priv , \id{i} }^{(j)}, P \right ) \stackrel{?}{=} e \left( Q_{\id{i}}, P_{pub}^{(j)} \right)
        \end{equation*}
        
        %If the check fails, report that PKG as malicious and request another PKG. 
        Next, $\id{i}$ calculates the private key $sk_{\id{i}}$ using the Lagrange coefficients $b_j$ as follows: 
        \begin{equation*}
            sk_{\id{i}} = \sum\limits_{j\in\Lambda} b_j Q_{priv,\id{i}}^{(j)} \quad \textrm{for} \quad b_j = \prod_{z\in \Lambda} \frac{z}{z-j}
        \end{equation*}
        \end{enumerate}
        In this way, no user or PKG learns the master key $sk_{msk}$ of the system. This algorithm combines \texttt{DKG.Reconstruct}, \texttt{IBE.Extract} and \texttt{BE.KeyGen} algorithms.

    \bigskip
    \bigskip
    \bigskip
    \bigskip
    \bigskip
    \bigskip
    \bigskip 
    \bigskip
    \bigskip
    \bigskip
    \bigskip
    \bigskip
    \bigskip
    \bigskip
    \bigskip 
    \bigskip
    \bigskip
    \bigskip
    \bigskip 
\end{description}
\end{algorithm}

\newpage
\thispagestyle{empty}
\begin{algorithm}[H]
\begin{description}
    \item[\texttt{Publish($params, \mathcal{S}, m$)}:] Takes the message $m$, the subset $\mathcal{S}$ of size $\eta$ and the public parameters $params$, output a broadcast message $\mathcal{B}$.

    \begin{enumerate}
        \item Generate a random symmetric session key $k \leftarrow \{ 0,1 \}^{l}$.
        \item Choose a random value $\rho \in \{ 0,1 \}^{l}$ and compute $r$ as a hash of concatenated values $r = H_3 \left( \{ \rho \parallel k \} \right)$
        \item For each recipient $\id{i} \in \mathcal{S}$, compute the ciphertext, running the \texttt{IBE.Encrypt} algorithm, as follows.
            \begin{equation*}
                w_i = \rho \oplus H_2 \left( g_{\id{i}}^r \right) \; \; \; \textrm{where} \; \; \; g_{\id{i}} = e \left( Q_{\id{i}}, P_{pub} \right) \in G_T
            \end{equation*}
        \item Let $w$ be a randomised concatenation, then the authenticated data $\mathcal{A}$ is computed as                                  
        \begin{equation*}
                \begin{array}{lcl}
                    \mathcal{A} & = & \{ \eta \parallel rP \parallel k \oplus H_3 \left( \rho \right) \parallel w_1 \parallel w_2 \parallel \ldots \parallel w_\eta \} \\
                    & = & \{ \eta \parallel U \parallel v \parallel w \} \; \; \; \textrm{for} \; \; \; w = \{ w_1 \parallel w_2 \parallel \ldots \parallel w_\eta \}
                \end{array} 
            \end{equation*}
            
        And $\mathcal{M}$ a concatenation of the intended recipient set $\mathcal{S}$ and the plaintext message $m$, such that $\mathcal{M} = \{ m \parallel \mathcal{S} \}$. (\texttt{BE.Encrypt})
    
        \item Apply authenticated symmetric encryption
        \begin{equation*}
            \left< c, t \right> \leftarrow \mathtt{E}_k(\mathcal{M},\mathcal{A})
        \end{equation*}
        \item The following message is then published in the OSN
        \begin{equation*}
            \mathcal{B} = \{ \mathcal{A} \parallel t \parallel c \}
        \end{equation*}
    \end{enumerate}
    \item[\texttt{Retrieve($params, sk_{\id{i}}, \mathcal{B}$)}:] on input of the broadcast message $\mathcal{B}$ and the private key $sk_{\id{i}}$ of user $\id{i}$, reconstruct the plaintext message $m$. This algorithm comprises the \texttt{\{IBE,BE\}.Decrypt} algorithms. For each $i \in \{  \}$ \\

    \begin{enumerate}
        \item Compute $w_i \oplus H_2 \left( e \left( sk_{\id{i}}, U \right) \right) = \rho$ for $sk_{\id{i}}$, and $v \oplus H_3 \{ \rho \} = k$ 
        \item Set $r = H_3 \left( \rho, k \right)$. Verify $U \stackrel{?}{=} rP$. If the check fails, try next $W_i$ and return to 1.
        \item Retrieve $\left< \mathcal{M}, t' \right> \leftarrow \mathtt{D}_k(c, \mathcal{A})$
        \item Verify whether $t' \stackrel{?}{=} t \in \mathcal{B} $, and return $m$. Otherwise return $\bot$. 
    \end{enumerate}
\end{description}
\end{algorithm}
\newpage

% Some hacky stuff to get the algorithm right
\makeatletter
\setlength{\headsep}{19.8738pt}
\makeatother

\paragraph{No key escrow and key validation} Key escrow is avoided by the DKG protocol included in both the \texttt{Setup} and \texttt{KeyGen} stages of the algorithm. 
The last check of the \texttt{KeyGen} step of Algorithm~\ref{alg:our_scheme} validates the correctness of a users' shares. However, this check does not ensure security against malicious PKGs since the Pedersen protocol~\cite{art:Pedersen91a} is insecure in the presence of malicious PKGs. That is, malicious PKGs can still affect the outcome of certain bits of the shared master secret key $sk_{msk}$ with non-negligible advantage. To circumvent these issues, the DKG scheme from Gennaro et al.~\cite{art:GennaroJKR07} should be implemented. However, since the scheme is developed in a threat model where PKGs are assumed trustworthy, this concern falls out of the scope of this thesis. Implementation of the DKG protocol from Gennaro et al.~\cite{art:GennaroJKR07} occurs similar to the scheme from Pedersen~\cite{art:Pedersen91a} since it relies on the same mathematical concepts. Therefore, adapting Algorithm~\ref{alg:our_scheme} to a more hostile DKG environment should be straightforward.

\paragraph{Limited key validity} For the sake of clarity, concatenation of expiration dates with public keys is not explicitly included in Algorithm~\ref{alg:our_scheme}. However, with the help of Algorithm~\ref{alg:map_to_date} this should be trivial since only the interpretation of the identifier symbol $\id{i}$ changes from a permanent identifier of a user to only a temporary identifier when concatenated with an expiration date.

The proposed solution can be used in any OSN that assigns unique public identifiers, such as usernames. Since the public keys are represented as strings, users are not required to upload keys to an additional third party server. Distributed key generation solves the key escrow issues that come with IBE solutions.

\section{Summary}
We started this chapter with the definition of a security model describing the current OSN situation. With the help of this model, we defined privacy threats in the model along with an adversary definition and assumptions on potential adversaries. In this framework, cryptographic objectives were set that defined the boundaries of our further design decisions. From these design decisions, it followed that a PKG had to be included as an additional entity the security model. After the presentation of an updated security model Algorithm~\ref{alg:our_scheme} was developed and evaluated. The result is a scheme that protects against the adversaries from our model and achieves the aforementioned design goals.

We further showed the feasibility of applying Algorithm~\ref{alg:our_scheme} to existing OSNs by effectively implementing it on an existing OSN. The results of our implementation process are further discussed in Chapter~5.

%%% Local Variables: 
%%% mode: latex
%%% TeX-master: "thesis"
%%% End: 

\chapter{Implementation}
\label{cha:n}

\section{Existing Solutions}

\section{Anonymous Identity-Based Broadcasting Implementation}

\subsection{Implemented Scheme}

\subsection{Data Structures}

\section{Distributed Key Generation Implementation}

\subsection{Implemented Scheme}

\subsection{Data Structures}

\section{Evaluation}

\section{Performance Analysis}

\section{Conclusion}

%%% Local Variables: 
%%% mode: latex
%%% TeX-master: "thesis"
%%% End: 

\chapter{Conclusion}
\label{cha:conclusion}
This last chapter gives an overview of the topics presented in this thesis. Furthermore, we summarise the limitations of our current solution, possible future work and in which other domains our techniques can bring added value.

Identity Based Encryption (IBE) provides desirable properties to construct mechanisms to deliver privacy in OSNs. The minimal additional architectural support and the increased ease of key management represent a major motivation to implement IBE in OSNs. We show that using secret sharing and multi-PKGs there is no need to have a single trusted party, assuming that at most $t-1$ of the PKGs are compromised. Furthermore, the multiple PKG infrastructure can be maintained by several organisations, motivated by increased privacy in OSNs. 

As a proof of concept, we have extended Scramble to rely on an IBE multi-PKG infrastructure for Facebook thereby demonstrating such extension presents a tolerable overhead to end-users. The result is a Firefox application that is more user friendly than previous alternatives since public keys are recognisable user \id{}s. In contrast to the earlier abstract notion of a public key stored and authenticated on a complex public key infrastructure, users can immediately relate owners to their public keys. 

Furthermore, the presented solution is practically applicable since it requires no changes to the current OSN infrastructure. This enables users to rely on our infrastructure even in OSN environments that are reluctant to support these forms of increased confidentiality.

In contrast to previous solutions, our infrastructure is immediately ready to use. Users are no longer required to subscribe to an additional third party infrastructure before being able to send encrypted messages. Therefore, it is possible to share content with users not holding private keys to their identity since the valid public key is directly represented by their \id{} in the OSN. This forces curious users to register if they wish to view the protected content shared with them. Privacy concerned users relying on cryptographic primitives are then no longer an isolated breed limited to communication with other privacy aware peers. Conversely, they serve as pioneers motivating other users in their environment to turn to similar solutions.

However, the presented implementation is only a proof of concept since it is currently only applicable as a Firefox extension on Facebook. We endeavour to obtain a full open source project that supports different browsers and OSNs. Furthermore, the server side implementation of our current solution is only simulated in a local environment. For use in distributed environments such as the internet, more advanced DKG protocols in the asynchronous setting should be considered. Another limitation of the achieved implementation is that it does not include an authentication mechanism with the OSN environment. Before translating our current mechanism to a practical environment, all these limitations should be resolved. However, responding to these issues is only a matter of implementation and not of deliberate design decisions. Therefore, the current implementation definitely lays the foundation for practical usable IBE on OSNs. 

There are some important open challenges that call for further research. Although current execution times achieved by our prototype are tolerable, techniques of randomness reuse possibly result in even higher performance gains. In addition, a more formal security discussion of our scheme is desirable and can be subject of future work.

We conclude this work by emphasising the applicability of our current scheme to other domains than broadcasting of messages on OSNs. The presented architecture can be applied to other media than text messages such as photos and videos. Consequently, the algorithm can find adoption in a wider set of OSNs like Youtube, Instagram and Snapchat. The proposed broadcasting scheme is also valuable for e-mail applications with multiple recipients. With the increasing influence of internet on our daily communication, the amount of broadcasting applications is even expected to increase. Consequently, although this thesis covers application of IBE in OSN environment, it implements the groundwork for many promising features an IBE scheme with multiple PKGs has to offer in future applications.


%%% Local Variables: 
%%% mode: latex
%%% TeX-master: "thesis"
%%% End: 


% If you have appendices:
\appendixpage*          % if wanted
\appendix
\chapter{Installing and Executing the Code}
\label{app:A}
Appendices hold useful data which is not essential to understand the work
done in the master thesis. An example is a (program) source.
An appendix can also have sections as well as figures and references\cite{h2g2}.

\section{Setting up the DKG}
\lipsum[50]

\section{Setting up Scramble}
\lipsum[15-17]

%%% Local Variables: 
%%% mode: latex
%%% TeX-master: "thesis"
%%% End: 

% ... and so on until
\include{app-n}

\backmatter
% The bibliography comes after the appendices.
% You can replace the standard "abbrv" bibliography style by another one.
\bibliographystyle{abbrv}
\bibliography{references}

\end{document}

%%% Local Variables: 
%%% mode: latex
%%% TeX-master: t
%%% End: 
