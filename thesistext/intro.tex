\chapter{Introduction}
\label{cha:intro}
The newest internet trend at the dawn of the 21st century certainly is the Online Social Network (OSN). Words like tweeting, sharing, liking, trending and tagging have found common acceptance in the vocabulary of today's internet savvy users while services like Facebook, Google+, LinkedIn and Twitter have become part of everyday life. 

The far reaching influence of today's most popular OSNs is best illustrated with the help of some statistics. In May 2013, 72\% of all internet users were active on a social network~\cite{site:Jones13}. At the time of writing, Facebook has 1.23 Billion monthly active users which corresponds to 17\% of the global population~\cite{site:Bullas14,site:worldometers}. Furthermore, the average Facebook user spends 15 hours and 33 minutes online per month~\cite{site:StatisticBrain}. These numbers show that social networks no longer represent the latest craze of an internet bubble but are conversely deeply rooted in our daily habits.

\section{Problem Statement}


\section{Existing Solutions}
\label{sec:existing_solutions}
Several existing solutions have been proposed in literature, all trying to solve most of the earlier mentioned issues in OSNs.

\paragraph{flyByNight~\cite{art:LucasB09}} is a Facebook application that protects user data by storing it in encrypted form on Facebook. It relies on Facebook servers for its key management and is therefore not secure against active attacks by Facebook itself.

\paragraph{NOYB (None Of Your Business)~\cite{art:GuhaSTF08}} replaces the details of user $A$ with those of random users $B$ and $C$ thereby making this process only reversible by friends who are allowed to see the profile of user $A$. However this can not be applied to user messages or status updates that are the most frequently used features in the OSNs considered in this thesis.

\paragraph{FaceCloak~\cite{art:LuoXH09}} stores published Facebook data on its servers in encrypted form and replaces the data on Facebook with random text fetched from Wikipedia. This could be a useful mechanism to prevent OSNs from blocking security aware users because they are scared to see their advertising revenues shrink. However, this approach has the disadvantage that other users could take this data to be genuine user content. Furthermore, FaceCloaks architecture leads to an inefficient key distribution system.

\paragraph{Persona~\cite{art:BadenBSBS09}} is a scheme that can be used as a Firefox extension to let users of an OSN determine their own privacy by supporting the ability to encrypt messages to a group of earlier defined friends based on \textit{attribute-based encryption (ABE)}\cite{art:SahaiW04}. The scheme supports lots of useful use cases such as sending messages to all friends that are related to a certain attribute or even encrypting messages to friends of friends. The major drawback of this system however is that every new friend has to exchange a public key before he is able to interact in the privacy preserving architecture consequently requiring an infrastructure for broadcasting and storing public keys. Furthermore, to support the encryption of messages to friends of friends, user defined groups should be made available publicly thereby making the public key distribution system even more complicated. Finally the proposed ABE encryption scheme is 100 to 1000 times slower than a standard RSA operation~\cite{art:BadenBSBS09}.

\paragraph{Scramble~\cite{art:BeatoKW11}} is a Firefox extension that allows defining groups of friends that are given access to certain social network updates. The tool uses public key encryption based on OpenPGP~\cite{rfc4880} to broadcast encrypted messages on almost any platform. Furthermore Scramble provides the implementation of a tiny link server such that OSN policies not allowing to post encrypted data are bypassed. However, as indicated by usability studies~\cite{art:WhittenT99} OpenPGP has a higher usage threshold because an average user does not manage to understand OpenPGP properly. Additionally, Scramble has to rely on the security decisions of the web of thrust. It therefore inherits the unpleasant property of OpenPGP that the user can not be sure that the used PGP key actually belongs to the intended Facebook profile.

The most unattractive property of all the above applications is that they have to rely on a rather complex infrastructure. Persona has to support an extended public key distribution system and Scramble relies on the leap-of-faith OpenPGP web of trust. All proposed solutions require users with no cryptographic background on asymmetric cryptography to make responsible decisions concerning the management of their keys. Furthermore, maintaining such complex key infrastructures gets more and more complex as more users subscribe.

\section{Goals of this Thesis}
\label{sec:goals_of_this_thesis}
The goal of this thesis is to develop an architecture that solves the issues discussed in Section~\ref{sec:problem_statement} thereby taking the challenges and pitfalls from earlier solutions in Section~\ref{sec:existing_solutions} into account. Specifically the architecture should have the following properties:
\begin{itemize}
 \item \textbf{User friendly:} An average OSN user should be able to use the resulting architecture, i.e. a user with no knowledge on cryptographic primitives.
 \item \textbf{Applicable:} The original OSN environment should not be altered since some OSN providers are probably not willing to support a more confidential architecture because it could possibly hurt their business model.
 \item \textbf{Immediately ready to use:} No additional registration or subscription to third party key architectures should be required to enable usage of the system. As soon as a user subscribes to the OSN provider he should be able to start receiving confidential messages.
\end{itemize}


\section{Structure of this Thesis}

%%% Local Variables: 
%%% mode: latex
%%% TeX-master: "thesis"
%%% End: 
