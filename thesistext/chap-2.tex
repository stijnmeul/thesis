\chapter{Literature Review}
\label{cha:2}

\section{Identity-Based Encryption}
Shamir~\cite{art:Shamir84} already proposed a concept of identity-based cryptography in 1984. In identity-based cryptography any string can be a valid public key for encryption or signature schemes thereby eliminating the need for digital certificates. Identity-based cryptography proves to be particularly elegant if the public key is related to an attribute that uniquely identifies the identity of the user like an e-mail address, an IP address or a telephone number. Consequently, identity-based cryptography reduces system complexity and the cost for establishing and managing the Public Key Infrastructure~(PKI)~\cite{art:BaekNSS04}. 


\subsection{Definition}
A generic Identity-Based Encryption (IBE) scheme is composed of four probabilistic polynomial time algorithms:
\begin{description}
    \item[\texttt{IBE.Setup($1^k$)}] On input of a security parameter $k$, outputs a master secret $s_k$ and public parameters $params$.
    \item[\texttt{IBE.Extract($params, s_k, \id{}$):}] Takes public parameters $params$, the master secret $s$, and an \id{} as input and returns the private key $s_{\id{}}$ corresponding to the identity \id{}.
    \item[\texttt{IBE.Encrypt($params, \id{}, m$):}] Returns the encryption $c$ of the message $m$ on the input of the public parameters $params$, the \id{}, and the arbitrary length message $m$.
    \item[\texttt{IBE.Decrypt($s_{\id{}}, c$):}] Decrypts the ciphertext $c =$ \texttt{IBE.Encrypt}($params, \id{}, m$) back to the message $m$ on input of the secret key $s_{\id{}}$ corresponding to the receiving identity \id{}.
\end{description}

\begin{figure}[ht]
    \begin{center}
    \scalebox{0.78}{
        \begin{tikzpicture}[auto, node distance=1mm, align=center,
            block/.style={rectangle,text width=6em,text centered,minimum height=11mm},
            line/.style={draw,very thick, ->},
            line2/.style={draw,very thick, <->},
            leg/.style={text centered},
            ]
            %\draw[help lines] (-6,-5) grid (8,3);
            \path
                % Images
                (-0.5,3) node [block] (pkg) {\includegraphics[scale=0.2]{img/pkg.png}}
                (-4,0) node [block] (alice) {\includegraphics[scale=0.2]{img/alice.png}}
                (4,0) node [block] (bob) {\includegraphics[scale=0.2]{img/bob.png}}
                % Text
                (-0.5, 0.5) node [leg] (test) {$c$}
                (2, 2) node [leg] (test2) {$s_{\id{Bob}}$}
                ;
                
       \node[below=of pkg] {\textbf{PKG}};
       \node[below=of alice] {\textbf{Alice} \\ 3. $c \leftarrow $\texttt{IBE.Encrypt($params, \id{Bob}, m$)}};
       \node[below=of bob] {\textbf{Bob} \\ 5. $m \leftarrow $\texttt{IBE.Decrypt($s_{\id{Bob}}, \id{Bob}$)}};
       \node[above=of pkg,align=left] {
               1. $\left< s_k, params \right> \leftarrow $\texttt{IBE.Setup($1^k$)} \\
               2. publish $params$ \\
               \\
               4. $s_{\id{Bob}} \leftarrow $\texttt{IBE.Extract($params, s_k, \id{Bob}$)}
               };
       \begin{scope}[every path/.style=line]
        \path (alice.east) -- (bob.west);         
       \end{scope}
       \begin{scope}[every path/.style=line2]
        \path (pkg.south east) -- (bob.north west);    
       \end{scope}


        \end{tikzpicture}
    }
    \end{center}
    \caption{Multiple $(n,t)$-PKG IBE for OSNs overview, for a message $m$ published for the set $\mathcal{S}$ for $t=3$.}
    \label{fig:overview}
\end{figure}


Although Shamir easily constructed an identity-based signature scheme based on RSA in 1984, the use case of IBE remained an open problem until the introduction of bilinear maps. In~\cite{art:BonehF01} Boneh and Franklin propose the first practically usable IBE scheme based on the Weil pairing. However, the security proof in~\cite{art:BonehF01} still relies on the random oracle assumption. Canetti et al.~\cite{art:CanettiHK03} succeed in proposing a secure IBE scheme without having to rely on the random oracle model. However, the attacker model in~\cite{art:CanettiHK03} requires the adversary to declare which identity it will target, therefore the scheme in~\cite{art:BonehF01} is considered more secure as attackers can adaptively choose the targeted identity. Gentry~\cite{GentryRandomOracles} proposes a more efficient alternative to this scheme without random oracles while achieving shorter public parameters. 

\subsection{Anonymous Identity-Based Encryption}

\section{Broadcast Encryption}

\subsection{Definition}

\subsection{Anonymous Broadcast Encryption}

\subsection{Outsider-Anonymous Broadcast Encryption}


\section{Secret Sharing}

\subsection{Definition}
\begin{defn}[Secret Sharing Scheme]
\label{def:secret_sharing_scheme}
 A \textit{Secret Sharing Scheme} is a cryptographic scheme that divides a secret $S$ into $n$ pieces of data $S_1, \ldots, S_n$ called \textit{shares}. Shares are distributed over $n$ different parties called \textit{shareholders} such that specific subsets of the distributed shares allow reconstruction of the original secret $S$.
\end{defn}

\begin{defn}[Threshold scheme]
\label{def:threshold_scheme}
 A $\left( t, n \right)$ \textit{threshold scheme} $\left( t \leq n \right)$ is a secret sharing scheme by which a trusted party securely distributes $n$ different shares $S_i$ to $n$ different parties $P_i$ for $1 \leq i \leq n$ such that any subset of $t$ or more different shares $S_i$ easily allows to reconstruct the original secret $S$. Knowledge of $t-1$ or less shares is insufficient to reconstruct the original secret $S$.
\end{defn}

\begin{defn}[Perfect threshold scheme]
\label{def:threshold_scheme}
 A $\left( t, n \right)$ threshold scheme is said to be \textit{perfect} if no subset of fewer than $t$ shareholders can derive any partial information in the information theoretic sense about the original secret $S$ even with infinite computational resources.
\end{defn}

\subsection{Shamir Secret Sharing}
In 1979, both Shamir~\cite{art:Shamir79} and Blakley~\cite{art:Blakley79} independently found an algorithm achieving perfect threshold secret sharing. Shamir's solution was based on polynomial interpolation while Blakley's algorithm relied on finite geometries. Blakley secret sharing uses more bits than necessary as it describes multidimensional planes. In contrast, Shamir secret sharing requires as many bits for each share as the length of the original secret. Therefore Shamir secret sharing has gained more popularity in both research communities and in practical implementations.

\begin{algorithm}
\caption{Shamir's $\left( t, n \right)$ threshold scheme~\cite{book:handbook_of_applied_cryptography} }
\label{alg:shamirs_threshold_sheme}
 \textbf{Goal}: A trusted party $T$ distributes shares of a secret $S$ to $n$ parties.
 
 \textbf{Result}: If a subset of at least $t$ out of $n$ shareholders collaborates, they can reconstruct the original secret $S$.
 \begin{enumerate}
  \item \textit{Setup} The trusted party T begins with a secret integer $S \geq 0$ it wishes to distribute among $n$ parties
   \begin{enumerate}
    \item T chooses a prime $p > \max \left( S, n \right)$ and defines $a_0 = S$
    \item $T$ selects $t-1$ random, independent coefficients $a_1, \ldots, a_{t-1}, 0 \leq a_j \leq p-1$ defining the random polynomial over $\mathcal{Z}_p$, $f \left( x \right) = \sum^{t-1}_{j=0} a_j x^j$
    \item $T$ computes $S_i = f \left( i \right) \bmod p, 1 \leq i \leq n$ and securely transfers the share $S_i$ to shareholder $P_i$, along with a public index $i$.
   \end{enumerate}
   \item \textit{Reconstruction} Any group of $t$ or more shareholders pool their shares. Their shares provide $t$ distinct points $\left( x, y \right) = \left( i, S_i \right)$ allowing computation of the coefficients $a_j, 1 \leq j \leq t-1$ of $f \left( x \right)$ by Lagrange interpolation. The secret is recovered by calculating
 \begin{equation*}
  f \left( 0 \right) = \sum^t_{i=1}y_i \prod_{1 \leq j \leq t, j \neq i} \frac{x_j}{x_j-x_i} = S
 \end{equation*}
 \end{enumerate}
\end{algorithm}

The idea behind Shamir secret sharing is elegant in its simplicity. Any polynomial $f \left( x \right)$ of degree $t-1$ is uniquely defined by $t$ points lying on the polynomial. For example, it is possible to draw only one straight line between 2 different coordinates, a quadratic is fully defined by 3 different coordinates and so on. If the trusted party randomly generates a polynomial of degree $t-1$ it suffices to securely distribute one of $n$ different coordinates on the curve to each party $P_i, 0 \leq i \leq n$. A subset of at least $t$ different shareholders has to collaborate in order to reconstruct the original polynomial by interpolation. For security reasons the polynomial $f \left( x \right)$ is calculated in a finite field modulo a large prime number $p$. The complete mechanism of Shamir's threshold scheme can be found in Algorithm~\ref{alg:shamirs_threshold_sheme}. The mechanism behind reconstruction in Algorithm~\ref{alg:shamirs_threshold_sheme} is explained because the coefficients of an unknown polynomial $f \left( x \right)$ of degree less than $t$, defined by points $\left( x_i, y_i \right), 1 \leq i \leq t$ are given by the Lagrange interpolation formula

\begin{equation*}
 f \left( x \right) = \sum^t_{i=1}y_i \prod_{1 \leq j \leq t, j \neq i} \frac{x-x_j}{x_i-x_j}
\end{equation*}
A proof of this formula is omitted but can be found in~\cite{site:proofwiki_lagrange}.

\subsection{Verifiable Secret Sharing}
Verifiable secret sharing~\cite{art:ChorGMA85} tries to ensure the participating parties that their received shares are consistent by providing a verification mechanism. This verification mechanism can either detect an unfair dealer during setup or participants submitting incorrect shares during the reconstruction phase. The first verifiable secret sharing schemes were \textit{interactive}, i.e. interaction between shareholders and the trusted party was required to verify their shares. In \textit{non-interactive verifiable secret sharing} only the trusted party is allowed to send messages to the future shareholders. Shareholders can not communicate with each other neither can they send messages back to the trusted party. Non-interactive verifiable secret sharing is preferred over interactive alternatives as their is no chance of shareholders accidentally leaking too much information.

Popular verifiable secret sharing schemes are Feldman's scheme~\cite{art:Feldman87} and Benaloh's scheme~\cite{art:Benaloh86a}. No further details are given as a basic notion of verifiable secret sharing suffices for the remainder of this text.

\section{Distributed Key Generation}

\section{Conclusion}
The final section of the chapter gives an overview of the important results of this chapter. This implies that the introductory chapter and the concluding chapter don't need a conclusion.

\lipsum[66]

%%% Local Variables: 
%%% mode: latex
%%% TeX-master: "thesis"
%%% End: 
