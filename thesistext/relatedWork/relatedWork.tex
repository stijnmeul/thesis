\documentclass[11pt]{article}

\usepackage{dsfont}
\usepackage[none]{hyphenat}
\usepackage{amsmath}
\usepackage{mathabx}
\usepackage{anysize}
\usepackage{color}

\marginsize{3.5cm}{3.5cm}{2cm}{2cm}
\setlength{\parindent}{0pt}

\begin{document}

\section*{Related Work}

\paragraph{Identity Based Encryption} The first concept of Identity-Based Encryption (IBE) was introduced by Shamir in 1984~\cite{DBLP:conf/crypto/Shamir84}. Although Shamir easily constructed an identity-based signature scheme based on RSA, the use case of IBE remained an open problem until the introduction of bilinear maps. In~\cite{BonehFranklinIBE} Boneh and Franklin propose the first practically usable IBE scheme based on the Weil pairing. However, the security proof in~\cite{BonehFranklinIBE} still relies on the random oracle assumption. Canetti et al.~\cite{Canetti} succeed in proposing a secure IBE scheme without having to rely on the random oracle model. However, the attacker model in~\cite{Canetti} requires the adversary to declare which identity it will target, therefore the scheme in~\cite{BonehBoyenRandomOracles} is considered more secure as attackers can adaptively choose the targeted identity. Gentry~\cite{GentryRandomOracles} proposes a more efficient alternative to this scheme without random oracles while achieving shorter public parameters. 

\paragraph{Broadcast Encryption} Fiat and Naor introduced the first concept of Broadcast Encryption (BE) in~\cite{FiatBE}. The implementation in~\cite{FiatBE} requires a ciphertext of size $O \left( t \log^2 t \log n \right)$ to be secure against $t$ colluding users. The first fully collusion resistant scheme was proposed in~\cite{NaorNaor} by Naor et al. thereby making the ciphertext size independent of the number of colluding users. Halevy and Shamir further reduce the required ciphertext length for collusion resistant schemes in~\cite{HalevyS02}. It is the first paper in a series of many (\cite{DodisF02}, \cite{GoodrichST04} and \cite{LewkoSW08}) that achieves a ciphertext size that is only dependent on the number of revoked users $O \left( r \right)$. Boneh, Gentry and Waters~\cite{BonehGW05} consider using bilinear maps to achieve constant size ciphertexts and $O \left( n \right)$ public keys.

\paragraph{Identity Based Broadcast Encryption} Sakai and Furukawa are the first to define a collusion resistant identity based broadcast encryption (IBBE) scheme in~\cite{SakaiF07}. Independently from~\cite{SakaiF07} Delerabl\'{e}e realises a similar identity based broadcast encryption scheme and claims to be the first as well in~\cite{DelerableeIBBE}.  The size of the public key in both~\cite{SakaiF07} and~\cite{DelerableeIBBE} is proportional to the maximum size of the intended set of recipients while realising short ciphertexts and private keys. Baek et al.~\cite{BaekIBEandBE} define an IBBE scheme that requires only one pairing computation. The scheme in~\cite{BaekIBEandBE} is proven secure under the random oracle assumption where the attacker ties himself to a selective-ID attack. Gentry and Waters achieve identity based broadcast encryption with sublinear ciphertexts in~\cite{GentryW08}. Their scheme is proven secure against a stronger notion of adaptive security where the attacker can adaptively alter its queries depending on earlier received information. Barbosa and Farshim~\cite{BarbosaIBEKEM} proposed an identity-based key encapsulation scheme for multiple parties which is an extension of \textit{mKEM} as considered by Smart~\cite{SmartMKEM} to the identity-based setting. An mKEM is a Key Encapsulation Mechanism which takes multiple public keys as input. An encrypted message under mKEM consists of an encapsulated session key $K$ and a symmetric encryption of the plaintext message $M$ under $K$.

\paragraph{Recipient Anonymous Broadcast Encryption} All earlier mentioned references describing BE require the intended set of recipients to be published to realise a higher efficiency. Barth, Boneh and Waters~\cite{BarthBonehWaters} are the first to design a BE scheme that takes the anonymity of the recipient into account. The proposed anonymous broadcast encryption (ANOBE) scheme imposes a linear dependency of the ciphertext on the number of recipients and can only be proven secure in the random oracle model. In~\cite{LibertANOBE} Libert et al., propose an alternative ANOBE scheme that is proven secure in the standard model. Both~\cite{BarthBonehWaters} and~\cite{LibertANOBE} propose a tag based system that allows efficient decryption at the cost of making the public master key linear dependent on the total number of users. Krzywiekci et al.~\cite{KrzywieckiKK06} propose a scheme that is proportional to the number of revoked users. In~\cite{YuRL10}, Yu et al. design an architecture that even hides the number of users in the recipient set. Fazio and Perera introduce the notion of outsider anonymous broadcast encryption in~\cite{FazioOutsiderANOBE}. The scheme relies on IBE to encode where a recipient is positioned in a publicly published tree to achieve sublinear ciphertexts. This construction allows sublinear ciphertexts while attaining recipient anonymity to all users that are outside the intended set of receivers.

\bibliographystyle{abbrv}
\bibliography{referencesToRelatedWork}

\end{document}
