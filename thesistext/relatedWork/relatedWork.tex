\documentclass[11pt]{article}

\usepackage{dsfont}
\usepackage[none]{hyphenat}
\usepackage{amsmath}
\usepackage{mathabx}
\usepackage{anysize}
\usepackage{color}

\marginsize{3.5cm}{3.5cm}{2cm}{2cm}
\setlength{\parindent}{0pt}

\begin{document}

\section*{Related Work}

\paragraph{Identity Based Encryption} The first concept of Identity-Based Encryption (IBE) was introduced by Shamir in 1984~\cite{DBLP:conf/crypto/Shamir84}. Although Shamir easily constructed an identity-based signature scheme based on RSA, the use case of IBE remained an open problem until the introduction of bilinear maps. In~\cite{BonehFranklinIBE} Boneh and Franklin propose the first practically usable IBE scheme based on the Weil pairing. However, the security proof in~\cite{BonehFranklinIBE} still relies on the random oracle assumption. Canetti et al.~\cite{Canetti} succeed in proposing a secure IBE scheme without having to rely on the random oracle model. However, the attacker model in~\cite{Canetti} requires the adversary to declare which identity it will target, therefore the scheme in~\cite{BonehBoyenRandomOracles} is considered more secure as attackers can adaptively choose the targeted identity. Gentry~\cite{GentryRandomOracles} proposes a more efficient alternative to this scheme without random oracles while achieving shorter public parameters. 

\paragraph{Broadcast Encryption}
\bibliographystyle{abbrv}
\bibliography{referencesToRelatedWork}

\end{document}
