\documentclass[11pt]{article}

\usepackage{dsfont}
\usepackage[none]{hyphenat}
\usepackage{amsmath}
\usepackage{mathabx}
\usepackage{anysize}
\usepackage{color}

\marginsize{3.5cm}{3.5cm}{2cm}{2cm}
\setlength{\parindent}{0pt}

\begin{document}

\section*{Anonymous Identity-Based Broadcast Encryption Scheme}
We let $\lambda$ be the security parameter given to the setup algorithm such that a security level of 256 bits is realised. We let $G$ be some BDH parameter generator. $S$ is the set of desired recipients $i$ with $i = 1,..,n$. Symmetric encryption and decryption is done using AES Galois Counter Mode  $GCM_{enc} \left( P, A, K, IV \right)$ and  $GCM_{dec} \left( P, A, K, IV \right)$ respectively.\\
\\
\textbf{\textit{Setup$\left( \lambda \right)$:}} Given a security parameter $\lambda \in \mathds{Z}^{+}$, the algorithm works as follows:
\begin{enumerate}
 \item Run $G$ on input $\lambda$ to generate a prime $q$, two groups $\mathds{G}_1, \mathds{G}_2$ of order $q$, and an admissible bilinear map $e: \mathds{G}_1 \times \mathds{G}_2 \rightarrow \mathds{G}_T$. Choose a random generator $P \in \mathds{G}_1$
 \item Pick a random $s \in \mathds{Z}^{*}_q$ and set $P_{pub} = sP$
 \item Choose a cryptographic hash function $H_1: \{ 0,1 \}^{*} \rightarrow \mathds{G}_1^{*}$ and $H_2: \mathds{G}_T^{*} \rightarrow \{ 0,1 \}^{256}$. The security analysis will view $H_1, H_2$ as random oracles.
\end{enumerate}

The symmetric key space is $K = \{ 0,1 \}^{256} = \{ K_1 || IV \}$ with $K_1 = \{ 0,1 \}^{128}$ and $IV = \{ 0,1 \}^{128}$. The ciphertext space is $C_i = \mathds{G}_1^{*} \times \{ 0,1 \}^{256}$. The system parameters are $params = \{ q, \mathds{G}_1, \mathds{G}_2, e, P, P_{pub}, H_1, H_2  \}$. The master key is $s \in \mathds{Z}_q^{*}$.\\
\\
\textbf{\textit{KeyGen$\left( \lambda,ID_i \right)$:}} For a given string $ID_i \in \{ 0,1 \}^{*}$ the algorithm does:
\begin{enumerate}
 \item Compute $Q_{\textrm{\tiny{ID}}_i} = H_1 \left( ID_{i} \right)$
 \item Set the private key $d_{\textrm{\tiny{ID}}_i}$ to be $d_{\textrm{\tiny{ID}}_i} = s Q_{\textrm{\tiny{ID}}_i}$ where $s$ is the master key.
 \item Return $d_{\textrm{\tiny{ID}}_i}$ to the corresponding user $ID_i$ over a secure channel.
\end{enumerate}

\textbf{\textit{Encrypt$\left( params, \lambda, K, S \right)$:}} To encrypt $K$ under the public keys $\{ ID_i \in \ S \}$:
\begin{enumerate}
 \item Generate a random symmetric session key $K_1 = \{ 0,1 \}^{128}$. Generate a random initialisation vector $IV = \{ 0,1 \}^{128}$ and set $K = \{ K_1 || IV \}$
 \item Choose a random $r \in \mathds{Z}_q^{*}$
 \item For each recipient $ID_i \in S, i = 1..n$, calculate the ciphertext
 \begin{equation*}
  C_i = K \oplus H_2 \left( g_{{\textrm{\tiny{ID}}_i}}^r \right) \; \; \; \textrm{where} \; \; \; g_{\textrm{\tiny{ID}}_i} = e \left( Q_{\textrm{\tiny{ID}}_i}, P_{pub} \right) \in \mathds{G}_T^{*}
 \end{equation*}
 
 \item Apply GCM with initialisation vector $IV$ and secret key $K_1$. Plaintext is set to $P_{text} = K$ and the additional authenticated data $A = \{ rP || C_1 || C_2 || .. || C_n \}$. GCM then outputs a ciphertext $C_T$ and an authentication tag $T$ such that
 \begin{equation*}
  \{ C_T, T \} = GCM_{enc} \left( P_{text}, A, K_1, IV \right)
 \end{equation*}
 
 \item The following message is then broadcasted over an insecure network
 \begin{equation*}
  M = \{ C_T || A || T\}
 \end{equation*}
\end{enumerate}

\newpage
\textbf{\textit{Decrypt$\left( params, d_{\textrm{\tiny{ID}}_i}, M \right)$:}} Parse broadcasted message $M$ as $\{ C_T || A || T\}$. For each $C_i \in A = \{ rP || C_1 || C_2 || .. || C_n \}$ do the following:
\begin{enumerate}
  \item Decrypt $C_i$ using the private key $d_{\textrm{\tiny{ID}}_i}$ by calculating
  \begin{equation*}
   C_i \oplus H_2 \left( e\left( d_{\textrm{\tiny{ID}}}, rP\right) \right) = K = \{ K_1 || IV \}
  \end{equation*}
  
 \item Decrypt $C_T$ by
  \begin{equation*}
   \{ P_{text}, T_{dec} \} = GCM_{dec} \left( C_T, A, K_1, IV \right)
  \end{equation*}
  
 \item Verify whether $T_{dec}$ corresponds to $T$ in $M$.\\ If $T \neq T_{dec}$, try next $C_i$. \\ When all $C_i$ are parsed ($i = n$) and still $T \neq T_{dec}$, return $\bot$.

\end{enumerate}

\end{document}